\section{Randwertaufgabe (RWA)}
\begin{itemize}
\item besteht aus 
	\begin{itemize}
	\item \textbf{Differentialgleichung (DGL)}\\
		\(\underline{D}(\underline{u}(\underline{x})) + \underline{\rho}(\underline{x}) = \underline{0} \qquad \forall x \in \Omega\)
	
	\item \textbf{Randbedingungen (RB)}\\
		\(\underline{D_1}(\underline{u}(\underline{x})) + \underline{r_1}(\underline{x}) = \underline{0} \qquad \forall x \in \Gamma_1\)
		
	\item \textbf{Differentialoperatoren}\\
		\(\underline{D}(...), (\underline{D_1}(...), (\underline{D_2}(...), ...\) Differentialoperatoren
	\item \(\underline{x}\) ... Ortsvektor
	\item \(\underline{\rho}, \underline{r_i}\) ... rechte Seite
	\item \(\Gamma_1\) ... Rand mit \underline{wesentlichen} Randbedingungen
	\item \(\Gamma_2\) ... Rand mit \underline{natürlichen} Randbedingungen\\
		\(\Gamma_1 \cup \Gamma_2 = \Gamma; \Gamma_1 \cap \Gamma_2 = \emptyset\)\\
		(für jede Koordinatenwichtung)
	\end{itemize}
	\item analytische Lösung nur für einfache Problemstellungen 
	\item Annahme: 
		\begin{itemize}
		\item kinematische Annahmen (Bernoulli)
		\item kinetische Annahmen
		\end{itemize}
\end{itemize}

