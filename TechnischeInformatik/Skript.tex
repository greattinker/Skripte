\documentclass[10pt,german,a4paper]{scrreprt}

\usepackage[ngerman]{babel}
\usepackage[utf8]{inputenc} 
\usepackage{lmodern}
\usepackage{listings}
\usepackage{color}
\usepackage{amsmath}
\usepackage{amssymb}
\usepackage{pdfpages}
\usepackage{geometry}
\usepackage{fancyhdr}
\usepackage{graphicx}
\usepackage{hyperref} 
\usepackage{float}		
\usepackage{relsize}		
\usepackage{euscript}	


\hypersetup{
	pdfborder=0
}

\geometry{a4paper,left=1cm,right=1cm, top=2cm, bottom=2cm} 

\def \Path {/media/Daten/Studium/Skripte/TechnischeInformatik}
\makeatletter
	\@addtoreset{chapter}{part}
\makeatother



\pagestyle{fancy}
\lhead{Skript Technische Informatik}
\chead{}
\rhead{\leftmark}
\lfoot{Christian Kroh \texttt{(s1428123)}}
\cfoot{}
\rfoot{\thepage}



\lstset{%
    columns=fullflexible,
    aboveskip=5pt,
    belowskip=10pt,
    basicstyle=\small\ttfamily,
    numbers=left,
    stepnumber=1, 
    numbersep=13pt,
    showspaces=false,
    showstringspaces=false,
    showtabs=false,
    xleftmargin=20pt,
    xrightmargin=10pt,
    framesep=5pt,
    framerule=3pt,
    frame=leftline,
    tabsize=2,
    breaklines=true,
    breakatwhitespace=true,
}

\renewcommand*{\chapterpagestyle}{fancy}

\newcommand{\insertrep}[1]{%
	\hspace*{-2.4cm}
	\fbox{\includegraphics[page=1,scale=0.8]{#1}}
	\includepdf[scale=0.8,pages=2-,frame]{#1}
}

\title{Einführung in die Technische Informatik\\Skript}
\subtitle{In der Hoffnung, dass es was nützt ...}
\author{Christian Kroh}
\date{\today{}, Dresden}

\begin{document}
\maketitle
\newpage
\tableofcontents
\newpage

\chapter{Allgemeine Einführung}
\section{Qualifikationsziele}
Die Studierenden kennen Systemarchitekturen und Modellierungsparadigmen
von VLSI-Systemen.\\
Sie sind in der Lage Beschreibungen von Hardware-Systemen durch
Simulation zu verifizieren und mithilfe typischer Werkzeuge in reale
Schaltungen umzuwandeln.\\
Sie können den Ressourcenbedarf, das Zeitverhalten und die Verlustleistung abschätzen oder evaluieren und daraus Entwurfsentscheidungen ableiten.
\section{Literatur}
\begin{itemize}
\item F. Kesel und R. Bartholomä: Entwurf von digitalen Schaltungen und
	Systemen mit HDLs und FPGAs, Oldenbourg Wissenschaftsverlag,
	ISBN 978-3-486-58976-4.
\item H.-D. Wuttke und K. Henke: Schaltsysteme – Eine automatenorientierte
	Einfürung, Pearson Studium, ISBN 3-8273-7035-3.
\item H. M. Lipp und J. Becker: Grundlagen der Digitaltechnik,
Oldenbourg Wissenschaftsverlag, ISBN 978-3-486-59747-9.

\end{itemize}



\newpage
\part{VLSI-Systementwurf}
\newpage
\chapter{Einführung}
\section{Themenschwerpunkte}
	\begin{description}

		\item[Verarbeitungsleistung]\hfill
			\\\textbf{Im Vordergrund stehen:}
			\begin{itemize}
				\item Schnelle Verarbeitung auch einzelner Bits
				\item Parallelität auf: \hfill
				\begin{itemize}
					\item Bitebene
					\item Befehlseben
					\item Threadebene
					\item Prozess- und Anwendungsebene
				\end{itemize}
				\item dynamische Rekonfiguration
			\end{itemize}
			$\Rightarrow$ Erfüllung der gegebenen Anforderungen

		\item[Systemintegration]\hfill
			\\\textbf{Im Vordergrund stehen:}
			\begin{itemize}
				\item Mehrprozessorsyteme, Mehrkern-, Vielkernprozessorsysteme
				\item Mehr-Chip- / Einzel-Chip-Lösungen (System-on-a-Chip)
				\item Parallele Entwicklung von HW und SW
	(HW-/SW-Codesign)
				\item System-Prototyping (FPGA-Entwurf)
			\end{itemize}
			$\Rightarrow$ Kosteneinsparung, Entwicklungszeiteinsparung (Time-to-Market)

		\item[Verlustleistung]
			\hfill\\\textbf{Im Vordergrund stehen:}
			\begin{itemize}
				\item Verlustleistung im Standby (Akkubetrieb)
				\item Maximales Abwärmebudget
			\end{itemize}

			\hfill\\\textbf{Kenngrößen:}
			\begin{itemize}
				\item Statische und dynamische Verlustleistung
				\item MIPS pro Watt
			\end{itemize}

			$\Rightarrow$ Sowohl für eingebettete Systeme als auch für Server


		\item[Korrektheit]
			\hfill\\\textbf{Aspekte:}
			\begin{itemize}
				\item Verifikation eines Schaltkreises, simulativ / formal
				\item Profiling und Debugging unter Echtzeitbedingungen (Trace)
			\end{itemize}
			$\Rightarrow$ Fehlerfreier Erstentwurf


		\item[Fehlertoleranz]\hfill
			\\\textbf{Toleranz gegenüber:}
			\begin{itemize}
				\item Permanenten Fehlern (zeitunabhängig nach erstem Auftreten)
				\item Intermitierenden Fehlern (nur unter bestimmten Betriebsbedingungen)
				\item Transienten Fehlern (aufgrund statischer Störungen)
			\end{itemize}

			\hfill\\\textbf{Fehlererkennung und -korrektur:}
			\begin{itemize}
				\item Autonom durch Hardwarearchitektur
				\item Aus Kombination von HW und SW
			\end{itemize}

			$\Rightarrow$ Insbesondere wichtig für Sicherheitskritische, hochverfügbare und langlebige zuverlässige Systeme
			\\$\Rightarrow$ Steigende Siginifikanz mit abnehmenden Strukturgrößen (Integrationsgrad) sowie steigender Transistoranzahl (Moore's Law)
	\end{description}


\section{Inhalte der Lehrveranstaltung}
\subsection{Inhalte der Vorlesung}
	\begin{enumerate}
		\item Klassifikation von Schaltkreisen
		\item Grundlagen des Schaltkreisentwurfs
		\item Automatendarstellung, -kopplung, -vereinfachung
		\item Hardwarebeschreibungssprachen
		\item Programmierabare Schaltkreise, insbesondere FPGAs - Teil 1
		\item Programmierabare Schaltkreise, insbesondere FPGAs - Teil 2
		\item Modellierung und Simulation
		\item Zeitverhalten und Test
		\item Hochgeschwindigkeit und Verlsustleistung
		\item Anwendungsbeispiele
	\end{enumerate}

\subsection{Inhalte des Praktikums}
	\begin{enumerate}
		\item Altera Quartus-Toolchain \& Praktikumsboard DE0 mit Cyclone-3
		\item Schaltnetze und Schaltwerke
		\item Modularisierung
		\item \grqq Komplexe\grqq Anwendung: Stoppuhr
	\end{enumerate}


\section{Klassifikation von ICs}

\subsection{Zwei Sichten}
	\hfill\\Klassifizierung von integrierten Schaltkreisen (ICs) in
	\begin{itemize}
		\item Standarschaltkreis (Standard-IC) und
		\item applikationspezifische Schaltkreise (Application-Specific IC, \textbf{ASIC})
	\end{itemize}
	unter zwei Gesichtspunkten möglich:
	\begin{itemize}
		\item \textbf{Herstellungssicht}
		\item \textbf{Entwurfssicht}
	\end{itemize}

	\paragraph{Herstellungssicht:}
	\begin{itemize}
		\item Standard-IC = große Stückzahl für viele Kunden
		\item ASIC = für eine/n Kunden/Applikation speziell entwickelter und gefertigter IC mit zugeschnittener Funktionalität
	\end{itemize}

	\paragraph{Entwurfssicht:}
	\begin{itemize}
		\item Standard-IC = (Hardware-)Funktionalität kann nicht vom Anwender beeinflusst werden
		\item ASIC = vom Anwender selbst entwickelte (Hardware-)Funktionalität
	\end{itemize}

	\newpage
	\begin{center}
	\includegraphics[page=17, width=0.7\linewidth, trim=30mm 20mm 20mm 40mm, clip]{\Path/resources/Vorlesung/VLSI/01_Einfuehrung.pdf}
	\\\includegraphics[page=18, width=0.7\linewidth, trim=30mm 20mm 20mm 40mm, clip]{\Path/resources/Vorlesung/VLSI/01_Einfuehrung.pdf}
	\\\includegraphics[page=19, width=0.7\linewidth, trim=30mm 20mm 20mm 40mm, clip]{\Path/resources/Vorlesung/VLSI/01_Einfuehrung.pdf}
	\end{center}

\subsection{Anwenderprogrammierbare IC (ASIC)}
	\paragraph{Merkmale:}
	\begin{itemize}
		\item Field-Programmable $\Leftrightarrow$ feldprogrammierbar
		\item Vor Ort (im Feld) vom Anwender programmierbar
		\item Hardware ist fix. Funktionalität kann aber mittels spezieller Konfiguration \grqq programmiert\grqq werden
	\end{itemize}
	\paragraph{Anwendung:}
	\begin{itemize}
		\item Anwendungsspezifische IC bei kleinen und mittleren Stückzahlen
		\item Mehrfach neu programmierbar zwecks Optimierung und Fehlerbehebung, auch während des praktischen Einsatzes
		\item Einfache Integration eines ganzen Systems auf einem Chip
		\item Prototyping, HW-/SW-Codesign
	\end{itemize}

\subsection{Hardwareprogrammierung}
	\begin{center}
	\includegraphics[page=21, width=0.7\linewidth, trim=30mm 20mm 20mm 40mm, clip]{\Path/resources/Vorlesung/VLSI/01_Einfuehrung.pdf}
	\end{center}
%\newpage
\subsection{Programmiertechnologien}
	\begin{center}
	\includegraphics[page=22, width=0.7\linewidth, trim=30mm 20mm 20mm 40mm, clip]{\Path/resources/Vorlesung/VLSI/01_Einfuehrung.pdf}
	\\\includegraphics[page=23, width=0.7\linewidth, trim=30mm 20mm 20mm 40mm, clip]{\Path/resources/Vorlesung/VLSI/01_Einfuehrung.pdf}
	\\\includegraphics[page=24, width=0.7\linewidth, trim=30mm 20mm 20mm 40mm, clip]{\Path/resources/Vorlesung/VLSI/01_Einfuehrung.pdf}
	\\\includegraphics[page=25, width=0.7\linewidth, trim=30mm 20mm 20mm 40mm, clip]{\Path/resources/Vorlesung/VLSI/01_Einfuehrung.pdf}
	\\\includegraphics[page=26, width=0.7\linewidth, trim=30mm 20mm 20mm 40mm, clip]{\Path/resources/Vorlesung/VLSI/01_Einfuehrung.pdf}
	\\\includegraphics[page=27, width=0.7\linewidth, trim=30mm 20mm 20mm 40mm, clip]{\Path/resources/Vorlesung/VLSI/01_Einfuehrung.pdf}
	\\\includegraphics[page=28, width=0.7\linewidth, trim=30mm 20mm 20mm 40mm, clip]{\Path/resources/Vorlesung/VLSI/01_Einfuehrung.pdf}
	\end{center}

\subsection{Klassifikation Anwenderprogrammierbare IC}
	\begin{center}
	\includegraphics[page=29, width=0.7\linewidth, trim=30mm 20mm 20mm 40mm, clip]{\Path/resources/Vorlesung/VLSI/01_Einfuehrung.pdf}
	\\\includegraphics[page=30, width=0.7\linewidth, trim=30mm 20mm 20mm 40mm, clip]{\Path/resources/Vorlesung/VLSI/01_Einfuehrung.pdf}
	\\\includegraphics[page=31, width=0.7\linewidth, trim=30mm 20mm 20mm 40mm, clip]{\Path/resources/Vorlesung/VLSI/01_Einfuehrung.pdf}
	\end{center}

\newpage

\chapter{Schaltkreisentwurf}

\section{Abstraktionsebenen und Sichten}
	\paragraph{Ebenen des Entwurfs:}
	\begin{itemize}
		\item Charakterisierung des jeweiligen Detailiertheitsgrades der Beschreibung des Entwurfsgrades
		\item Abstraktionsgrad von der eigentlichen physikalischen Realisierung
		\item Abstraktionsniveaus, Hierarchien
	\end{itemize}

	\paragraph{Sichten des Entwurfs:}
	\begin{itemize}
		\item Betrachtung des Entwurfsgegenstandes aus verschiedenen Richtungen
		\item Sicht = Eigenschaften die den Entwurfsgegenstand Charakterisieren
		\item Alle Sichten auf jeder Entwurfseben $\Rightarrow$ Y-Diagramm / x-Diagramm
	\end{itemize}

\subsection{Abstraktionsebenen}
	\paragraph{Systemebene:} Systemkonzept des Entwurfsgegenstandes
	\paragraph{Algorithmische Ebene:} Algorithmische Beschreibung des Entwurfsgegenstandes
	\paragraph{Register-Transfer Ebene:} Datentransfer und -verarbeitung zwischen Registern
	\paragraph{Logikebene:} Beschreibung auf Gatterniveau
	\paragraph{Schaltkreiseben:} Transistorebene im weiteren Sinne, umfasst:
		\begin{itemize}
			\item Schalterebene
			\item Schaltungsebene
			\item Bauelementebene
			\item Technologieebene
		\end{itemize}

\subsection{Sichten}
	\paragraph{Verhaltenssicht:} Beschreibung des zeitlichen Verhaltens durch charaktersierende Variablen und deren Werteverläufe über die Zeit
		
		\begin{center}
			\boldmath\( \mathlarger{\vec{y}(t) = f(\vec{x}(t))} \)
		\end{center}
	\paragraph{Struktursicht:} Spezifizierung eines Objektes durch Subobjekte und deren Verbindungsstrukturen
	
	\paragraph{Geometriesicht:} Räumliche Ausdehnung der Subobjekte
	
	\paragraph{Testsicht:} Existenz oder Nichtexistenz angenommener struktureller oder funktionaler Defekte \\(F. J. Rammig, Systematischer Entwurf digitaler Systeme, B.G Teubner Stuttgart 1989)
	
\subsection{Y-Diagramm nach Gajski}
	\begin{itemize}
		\item Stellt Ebenen und Sichten in FOrm eines Y-Diagrammes dar
		\item Ursprünglich nur 3 Ebenen
		\item Heute: Erweiterung auf 5 Ebenen:
			\begin{itemize}
				\item Systemebene
				\item Algorithmische Ebene
				\item Register-Transfer Ebene
				\item Logikeben
				\item Schaltkreisebene
			\end{itemize}
	\end{itemize}
	\begin{center}
		\includegraphics[page=9, width=0.8\linewidth, trim=30mm 20mm 20mm 40mm, clip]{\Path/resources/Vorlesung/VLSI/02_Schaltkreisentwurf.pdf}
		\\\includegraphics[page=10, width=0.8\linewidth, trim=30mm 20mm 20mm 40mm, clip]{\Path/resources/Vorlesung/VLSI/02_Schaltkreisentwurf.pdf}
	\end{center}
	
\newpage
\section{Entwurfsablauf}
	\paragraph{Allgemein:} Transformation einer Aufgabenstellung (Pflichtenheft) in einen fertigen Schaltkreis
	\paragraph{Top-Down-Strategie:}
		\begin{itemize}
			\item Systemebene $\rightarrow$ Schaltkreisebene
			\item Vorteil: Parallele Entwicklung auf unteren Ebenen
			\item Nachteil: Systemspezifikation zu Projektbeginn oft zu ungenau
		\end{itemize}
	\paragraph{Bottom-Up-Strategie:}
		\begin{itemize}
			\item Analyse vorhandener Komponenten
			\item Zusammensetzen von neuen Komponenten auf höherer Ebene im Sinne der Aufgabenstellung
			\item Nachteil: globales Ziel wird nicht immer erreicht
		\end{itemize}
	\paragraph{$\Rightarrow$ Meet-in-the-Middle}
	
	\paragraph{Entwurfsschritt:}
		\begin{itemize}
			\item generierende Aktivität
			\item überprüfende Aktivität
		\end{itemize}
	\paragraph{Syntheseschritt:}
		\begin{itemize}
			\item Abbildung eines Entwurfsschrittes in Richtung auf das Entwurfsziel
			\item Abstraktionsgrad sinkt, Detailhiertheitsgrad steigt
			\item Einbringung neuer Informationen
		\end{itemize}
	\paragraph{Analyseschritt:}
		\begin{itemize}
			\item Abbildung eines Entwurfsschrittes in umgekehrter Richtung zum Syntheseschritt
			\item Gewinnung abstrakter Informationen durch Zusammenfassen und Generalisieren von Details (Extraktionsprozess)
			\item Beispiel: Validierung eines Syntheseschrittes
		\end{itemize}
		
	\begin{center}
		\includegraphics[page=13, width=0.39\linewidth, trim=40mm 30mm 40mm 50mm, clip]{\Path/resources/Vorlesung/VLSI/02_Schaltkreisentwurf.pdf}
		\includegraphics[page=14, width=0.6\linewidth, trim=0mm 30mm 0mm 45mm, clip]{\Path/resources/Vorlesung/VLSI/02_Schaltkreisentwurf.pdf}
	\end{center}
	
\newpage
\section{Entwurfsstile}
	\begin{center}
		\includegraphics[page=15, width=0.8\linewidth, trim=40mm 28mm 40mm 60mm, clip]{\Path/resources/Vorlesung/VLSI/02_Schaltkreisentwurf.pdf}
	\end{center}
	
	\subsubsection{Gegenüberstellung Entwurfsalternativen}
		\begin{center}
			\includegraphics[page=16, width=0.8\linewidth, trim=35mm 28mm 30mm 60mm, clip]{\Path/resources/Vorlesung/VLSI/02_Schaltkreisentwurf.pdf}
		\end{center}

\subsection{Full-Custom Entwurf}
	\begin{center}
		\includegraphics[page=17, width=0.8\linewidth, trim=40mm 28mm 40mm 60mm, clip]{\Path/resources/Vorlesung/VLSI/02_Schaltkreisentwurf.pdf}
	\end{center}
	\paragraph{Merkmale:}
	\begin{itemize}
		\item Platzierung und Verdrahtung selbst entworfener Transistoren \& Gatter
		\item Auch Mischung von Analog- und Digitaltechnik, z.B. für spezielle I/O-Signaltreiber
		\item Erfüllung spezieller Anforderungen, z.B. gehärtet gegen Strahlung
		\item Häufig nur auf kleine Teilschaltungen angewendet
		\item Hoch qualifizierte Entwicklungsingenieure mit Detailkenntnissen zu den Prozessen erforderlich
	\end{itemize}
	\paragraph{Anwendung:}
	\begin{itemize}
		\item Sensorik, Mixed-Signal-Schaltungen
		\item Raumfahrt
	\end{itemize}
	\subsubsection{Beispiel: Inverter}
		\begin{center}
			\includegraphics[page=19, width=0.8\linewidth, trim=40mm 28mm 40mm 60mm, clip]{\Path/resources/Vorlesung/VLSI/02_Schaltkreisentwurf.pdf}
		\end{center}
	
\newpage
\subsection{Standardzellenentwurf}
	\begin{center}
		\includegraphics[page=20, width=0.8\linewidth, trim=40mm 28mm 40mm 60mm, clip]{\Path/resources/Vorlesung/VLSI/02_Schaltkreisentwurf.pdf}
	\end{center}
	\paragraph{Merkmale:}
	\begin{itemize}
		\item Platzierung und Verdrahtung vorgegebener Gatter- oder Makrozellen
		\item Auswahl einer Technologiebibliothek nach:
			\begin{itemize}
				\item Fertigungstechnologie, Strukturbreite und Funktion
				\item High-Speed, Low-Leakage oder Mischung aus Beidem
			\end{itemize}
		\item Werkzeuggestützte Platzierung und Verdrahtung
		\item Makrozellen:
			\begin{itemize}
				\item Vordefiniert oder per Generator kundenspezifisch erzeugt
				\item Bsp: RAM, ALU, I/O-Komponenten
			\end{itemize}
	\end{itemize}
	\paragraph{Anwendung:}
	\begin{itemize}
		\item Anwendungsspezifische Schaltkreise mit hohen Stückzahlen
		\item Starke Optimierung bzgl. Chipfläche, Geschwindigkeit und Verlustleistung
	\end{itemize}
	\subsubsection{Konventionelle Architektur vs. Strukturierte Architektur}
		\begin{center}
			\includegraphics[page=22, width=0.45\linewidth, trim=60mm 28mm 60mm 60mm, clip]{\Path/resources/Vorlesung/VLSI/02_Schaltkreisentwurf.pdf}
			\includegraphics[page=23, width=0.45\linewidth, trim=60mm 28mm 60mm 60mm, clip]{\Path/resources/Vorlesung/VLSI/02_Schaltkreisentwurf.pdf}
		\end{center}
	
\newpage
\subsection{Maskenprogrammierbare Gate Arrays}
	\begin{center}
		\includegraphics[page=27, width=0.8\linewidth, trim=40mm 28mm 40mm 60mm, clip]{\Path/resources/Vorlesung/VLSI/02_Schaltkreisentwurf.pdf}
	\end{center}
	\paragraph{Merkmale:}
	\begin{itemize}
		\item Festlegung der Funktion mittels Masken bei der Halbleiterfertigung
		\item Ausgewählte Masken: teilweise vorgefertigte IC (Master)
		\item Beispiele:
			\begin{itemize}
				\item MPGA (Maskprogrammable Gate-Array): auch MGA
					\begin{itemize}
						\item Vorgefertigte universelle Gatter/Makros mit fester Anordnung
						\item Verdrahtung kundenspezifisch
					\end{itemize}
				\item MROM: ROM dessen Inhalt vom Kunden mit Masken festgelegt wird
			\end{itemize}
	\end{itemize}
	\paragraph{Anwendung:}
	\begin{itemize}
		\item Anwendungsspezifische Schaltkreise bei mittleren Stückzahlen
		\item Kostenoptimiert mit reduzierten Optimierungspotential
	\end{itemize}
	\subsubsection{MPGA}
		\begin{center}
			\includegraphics[page=29, width=0.45\linewidth, trim=60mm 28mm 60mm 60mm, clip]{\Path/resources/Vorlesung/VLSI/02_Schaltkreisentwurf.pdf}
		\end{center}
	
	
\newpage
\subsection{Anwenderprogrammierbare IC}
	\begin{center}
		\includegraphics[page=30, width=0.8\linewidth, trim=40mm 28mm 40mm 60mm, clip]{\Path/resources/Vorlesung/VLSI/02_Schaltkreisentwurf.pdf}
	\end{center}
	\paragraph{Merkmale:}
	\begin{itemize}		
		\item Field-Programmable $\Leftrightarrow$ feldprogrammierbar
		\item Vor Ort (im Feld) vom Anwender programmierbar
		\item Hardware ist fix. Funktionalität kann aber mittels spezieller Konfiguration \grqq programmiert\grqq werden
	\end{itemize}
	\paragraph{Anwendung:}
	\begin{itemize}
		\item Anwendungsspezifische IC bei kleinen und mittleren Stückzahlen
		\item Mehrfach neu programmierbar zwecks Optimierung und Fehlerbehebung, auch während des praktischen Einsatzes
		\item Einfache Integration eines ganzen Systems auf einem Chip
		\item Prototyping, HW-/SW-Codesign
	\end{itemize}
	\subsubsection{Hardwareprogrammierung}
		\begin{center}
		\includegraphics[page=21, width=0.7\linewidth, trim=30mm 20mm 20mm 40mm, clip]{\Path/resources/Vorlesung/VLSI/01_Einfuehrung.pdf}
		\end{center}
	\subsubsection{Klassifikation Anwenderprogrammierbare IC}
		\begin{center}
			\includegraphics[page=33, width=0.8\linewidth, trim=35mm 28mm 30mm 58mm, clip]{\Path/resources/Vorlesung/VLSI/02_Schaltkreisentwurf.pdf}
		\end{center}
	\subsubsection{CPLD}
		Globale Vernetzung einer kleiner Anzahl von Funktionsblöcken (FB)
		\begin{center}
			\includegraphics[page=34, width=0.8\linewidth, trim=35mm 23mm 30mm 70mm, clip]{\Path/resources/Vorlesung/VLSI/02_Schaltkreisentwurf.pdf}
		\end{center}
	\subsubsection{CPLD-Funktionsblock}
		Funktionsblock bestehend aus PLA (Und/Oder-Matrix) und Makrozellen (MC)
		\begin{center}
			\includegraphics[page=35, width=0.5\linewidth, trim=35mm 23mm 30mm 70mm, clip]{\Path/resources/Vorlesung/VLSI/02_Schaltkreisentwurf.pdf}
		\end{center}
	\subsubsection{PLA}
		\begin{center}
			\includegraphics[page=36, width=0.8\linewidth, trim=35mm 23mm 30mm 40mm, clip]{\Path/resources/Vorlesung/VLSI/02_Schaltkreisentwurf.pdf}
		\end{center}
	\subsubsection{Makrozellen + I/O-Block}
		Verschiedene Betriebsspannungen für digitale Lokig (Core) und I/O-Pads
		\begin{center}
			\includegraphics[page=37, width=0.8\linewidth, trim=35mm 23mm 30mm 70mm, clip]{\Path/resources/Vorlesung/VLSI/02_Schaltkreisentwurf.pdf}
		\end{center}
		\paragraph{Konfiguration der Makrozelle:}	
		\begin{center}
			\includegraphics[page=38, width=0.8\linewidth, trim=35mm 93mm 30mm 70mm, clip]{\Path/resources/Vorlesung/VLSI/02_Schaltkreisentwurf.pdf}
		\end{center}
		\paragraph{Steuerung des I/O-Blocks:}\hfill\\
		Umschaltung des I/O-Pins zwischen Ein- und Ausgang (Tri-State) zur Laufzeit mittels seperatem Produktterm möglich.
	
	\subsubsection{FPGA-Architektur}	
		\begin{center}
			\includegraphics[page=39, width=0.8\linewidth, trim=35mm 30mm 20mm 58mm, clip]{\Path/resources/Vorlesung/VLSI/02_Schaltkreisentwurf.pdf}
		\end{center}
	
\section{Entwurfswerkzeuge}
	\begin{center}
		\includegraphics[page=40, width=0.8\linewidth, trim=35mm 87mm 30mm 65mm, clip]{\Path/resources/Vorlesung/VLSI/02_Schaltkreisentwurf.pdf}
	\end{center}
	Auswahl CAD-Werkzeuge:
	\begin{itemize}
		\item Cadence
		\item Xilinx ISE
		\item Altera Quartus
		\item Synopsys
	\end{itemize}
	
	
	
	
	

\chapter{Automaten}

\section{Automatendarstellung}

\subsection{Betrachtungsweisen}
	\paragraph{Verschiedene Sichten/Semantik:}
	\begin{itemize}
		\item Zustandsübergangsdiagramm (state diagram)
			\begin{itemize}
				\item Vernetzung von Zuständen
				\item Beispiele: UML-Zustandsdiagramm, Automatengrpahen, SM Charts, GRAFCET, Sequential Function Charts (SFC)
			\end{itemize}
		\item Ablaufdiagramm (flow chart)
			\begin{itemize}
				\item Vernetzung von Prozessen
				\item Zustand ergibt sich aus der Verkettung aller Variablenzustände
				\item Beispiele: UML-Aktivitätsdiagramm, Programmablaufplan (PAP)
			\end{itemize}
	\end{itemize}
	
	\subsubsection{Binärzähler mod 4 mit Trigger X und Übertrag Y}	
		\begin{center}
			\includegraphics[page=4, width=0.6\linewidth, trim=35mm 27mm 30mm 58mm, clip]{\Path/resources/Vorlesung/VLSI/03_Automaten.pdf}
		\end{center}
	
\subsection{Automatengraphen}	
	\begin{center}
		\includegraphics[page=5, width=0.8\linewidth, trim=35mm 67mm 30mm 58mm, clip]{\Path/resources/Vorlesung/VLSI/03_Automaten.pdf}
	\end{center}
	\paragraph{Endlicher Automat:} Menge der möglichen Eingabezeichen, Ausgabezeichen und inneren Zustände ist endlich ...
	\paragraph{Synchron getakteter Automat:} Zustandsübergänge aller Speicherglieder erfolgen gleichzeitig, synchron zu einem Taktsignal
	
	\subsubsection{Notation}
		\begin{center}
			\includegraphics[page=6, width=0.65\linewidth, trim=35mm 37mm 30mm 58mm, clip]{\Path/resources/Vorlesung/VLSI/03_Automaten.pdf}
		\end{center}
	\subsubsection{Grafische Darstellung}
		\begin{center}
			\includegraphics[page=7, width=0.65\linewidth, trim=35mm 39mm 30mm 58mm, clip]{\Path/resources/Vorlesung/VLSI/03_Automaten.pdf}
		\end{center}
		\paragraph{$\Rightarrow$ Prüfung auf:} Vollständigkeit und Widerspruchfreiheit
	
	\subsubsection{Getaktete Automaten}
		\paragraph{Theoretische Informatik}
		\begin{itemize}
			\item Verarbeitung von EIngabezeichen sofern vorhanden
			\item Jedes Zeichen in der Eingabe wird einmalig verarbeitet
		\end{itemize} 
		\paragraph{Technische Informatik}\hfill\\
		Taktung des Automaten mit einem Taktsignal $\rightarrow$ Abtatstung der Eingabe\\
		Konsequenzen:
		\begin{itemize}
			\item Zeichen für \grqq keine Eingabe \grqq erforderlich
			\item Abtastung \grqq derselben Eingabe \grqq in aufeinanderfolgenden Takten möglich
		\end{itemize} 
		$\Rightarrow$ Korrekte Modellierung des Taktsignals erforderlich
	
		\paragraph{Beispiel - Taktzustandsgesteuertes D-Flip-Flop (Latch)}\hfill\\
		(Taktsignal C als Eingabe) 
		\begin{center}
			\includegraphics[page=9, width=0.65\linewidth, trim=35mm 50mm 30mm 80mm, clip]{\Path/resources/Vorlesung/VLSI/03_Automaten.pdf}
		\end{center}
	
		\paragraph{Beispiel - Taktflankengesteuertes D-Flip-Flop}\hfill\\
		(Taktsignal C als Eingabe) 
		\begin{center}
			\includegraphics[page=10, width=0.65\linewidth, trim=35mm 26mm 30mm 80mm, clip]{\Path/resources/Vorlesung/VLSI/03_Automaten.pdf}
		\end{center}
	
		\paragraph{Beispiel - Taktflankengesteuertes D-Flip-Flop}\hfill\\
		(Definition: Zustandsübergang nur bei taktflanke) 
		\begin{center}
			\includegraphics[page=11, width=0.65\linewidth, trim=35mm 50mm 30mm 80mm, clip]{\Path/resources/Vorlesung/VLSI/03_Automaten.pdf}
		\end{center}
	
\subsection{SM-Charts}
		\begin{center}
			\includegraphics[page=12, width=0.65\linewidth, trim=35mm 30mm 30mm 60mm, clip]{\Path/resources/Vorlesung/VLSI/03_Automaten.pdf}
		\end{center}
		
		\subsubsection{Verzweigungen}
			\paragraph{Weitere Merkmale:}
			\begin{itemize}
				\item Selbstdefinierte Abkürzungen für komplexe Ausdrücke üblich
			\end{itemize}
	

\chapter{Hardwarebeschreibungssprachen - Hardware Description Language (HDL)}

\section{Allgemein}

\section{VHDL}

\section{Verilog}

\chapter{Field-programmable Gate-Array (FPGA)}

\section{Architektur}

\section{Funktionsblöcke}

\section{I/O-Zellen}

\section{Verdrahtung}
\subsection{Topologie}
\subsection{Technologie}

\section{Speicherelemente}
\subsection{LUT-RAM}
\subsection{Block-RAM}

\section{IP-Cores}

\section{Konfigurierbarkeit}

\section{Konfigurationsmodi}

\input{\Path/VLSI/07-Modellierung_Simulation}
\input{\Path/VLSI/08-Zeitverhalten_Test}
\input{\Path/VLSI/09-Hochgeschwindigkeit_Verlustleistung}


\part{Entwurf eingebetteter Systeme}
\newpage
%\chapter{Einführung}
\section{Themenschwerpunkte}
	\begin{description}

		\item[Verarbeitungsleistung]\hfill
			\\\textbf{Im Vordergrund stehen:}
			\begin{itemize}
				\item Schnelle Verarbeitung auch einzelner Bits
				\item Parallelität auf: \hfill
				\begin{itemize}
					\item Bitebene
					\item Befehlseben
					\item Threadebene
					\item Prozess- und Anwendungsebene
				\end{itemize}
				\item dynamische Rekonfiguration
			\end{itemize}
			$\Rightarrow$ Erfüllung der gegebenen Anforderungen

		\item[Systemintegration]\hfill
			\\\textbf{Im Vordergrund stehen:}
			\begin{itemize}
				\item Mehrprozessorsyteme, Mehrkern-, Vielkernprozessorsysteme
				\item Mehr-Chip- / Einzel-Chip-Lösungen (System-on-a-Chip)
				\item Parallele Entwicklung von HW und SW
	(HW-/SW-Codesign)
				\item System-Prototyping (FPGA-Entwurf)
			\end{itemize}
			$\Rightarrow$ Kosteneinsparung, Entwicklungszeiteinsparung (Time-to-Market)

		\item[Verlustleistung]
			\hfill\\\textbf{Im Vordergrund stehen:}
			\begin{itemize}
				\item Verlustleistung im Standby (Akkubetrieb)
				\item Maximales Abwärmebudget
			\end{itemize}

			\hfill\\\textbf{Kenngrößen:}
			\begin{itemize}
				\item Statische und dynamische Verlustleistung
				\item MIPS pro Watt
			\end{itemize}

			$\Rightarrow$ Sowohl für eingebettete Systeme als auch für Server


		\item[Korrektheit]
			\hfill\\\textbf{Aspekte:}
			\begin{itemize}
				\item Verifikation eines Schaltkreises, simulativ / formal
				\item Profiling und Debugging unter Echtzeitbedingungen (Trace)
			\end{itemize}
			$\Rightarrow$ Fehlerfreier Erstentwurf


		\item[Fehlertoleranz]\hfill
			\\\textbf{Toleranz gegenüber:}
			\begin{itemize}
				\item Permanenten Fehlern (zeitunabhängig nach erstem Auftreten)
				\item Intermitierenden Fehlern (nur unter bestimmten Betriebsbedingungen)
				\item Transienten Fehlern (aufgrund statischer Störungen)
			\end{itemize}

			\hfill\\\textbf{Fehlererkennung und -korrektur:}
			\begin{itemize}
				\item Autonom durch Hardwarearchitektur
				\item Aus Kombination von HW und SW
			\end{itemize}

			$\Rightarrow$ Insbesondere wichtig für Sicherheitskritische, hochverfügbare und langlebige zuverlässige Systeme
			\\$\Rightarrow$ Steigende Siginifikanz mit abnehmenden Strukturgrößen (Integrationsgrad) sowie steigender Transistoranzahl (Moore's Law)
	\end{description}


\section{Inhalte der Lehrveranstaltung}
\subsection{Inhalte der Vorlesung}
	\begin{enumerate}
		\item Klassifikation von Schaltkreisen
		\item Grundlagen des Schaltkreisentwurfs
		\item Automatendarstellung, -kopplung, -vereinfachung
		\item Hardwarebeschreibungssprachen
		\item Programmierabare Schaltkreise, insbesondere FPGAs - Teil 1
		\item Programmierabare Schaltkreise, insbesondere FPGAs - Teil 2
		\item Modellierung und Simulation
		\item Zeitverhalten und Test
		\item Hochgeschwindigkeit und Verlsustleistung
		\item Anwendungsbeispiele
	\end{enumerate}

\subsection{Inhalte des Praktikums}
	\begin{enumerate}
		\item Altera Quartus-Toolchain \& Praktikumsboard DE0 mit Cyclone-3
		\item Schaltnetze und Schaltwerke
		\item Modularisierung
		\item \grqq Komplexe\grqq Anwendung: Stoppuhr
	\end{enumerate}


\section{Klassifikation von ICs}

\subsection{Zwei Sichten}
	\hfill\\Klassifizierung von integrierten Schaltkreisen (ICs) in
	\begin{itemize}
		\item Standarschaltkreis (Standard-IC) und
		\item applikationspezifische Schaltkreise (Application-Specific IC, \textbf{ASIC})
	\end{itemize}
	unter zwei Gesichtspunkten möglich:
	\begin{itemize}
		\item \textbf{Herstellungssicht}
		\item \textbf{Entwurfssicht}
	\end{itemize}

	\paragraph{Herstellungssicht:}
	\begin{itemize}
		\item Standard-IC = große Stückzahl für viele Kunden
		\item ASIC = für eine/n Kunden/Applikation speziell entwickelter und gefertigter IC mit zugeschnittener Funktionalität
	\end{itemize}

	\paragraph{Entwurfssicht:}
	\begin{itemize}
		\item Standard-IC = (Hardware-)Funktionalität kann nicht vom Anwender beeinflusst werden
		\item ASIC = vom Anwender selbst entwickelte (Hardware-)Funktionalität
	\end{itemize}

	\newpage
	\begin{center}
	\includegraphics[page=17, width=0.7\linewidth, trim=30mm 20mm 20mm 40mm, clip]{\Path/resources/Vorlesung/VLSI/01_Einfuehrung.pdf}
	\\\includegraphics[page=18, width=0.7\linewidth, trim=30mm 20mm 20mm 40mm, clip]{\Path/resources/Vorlesung/VLSI/01_Einfuehrung.pdf}
	\\\includegraphics[page=19, width=0.7\linewidth, trim=30mm 20mm 20mm 40mm, clip]{\Path/resources/Vorlesung/VLSI/01_Einfuehrung.pdf}
	\end{center}

\subsection{Anwenderprogrammierbare IC (ASIC)}
	\paragraph{Merkmale:}
	\begin{itemize}
		\item Field-Programmable $\Leftrightarrow$ feldprogrammierbar
		\item Vor Ort (im Feld) vom Anwender programmierbar
		\item Hardware ist fix. Funktionalität kann aber mittels spezieller Konfiguration \grqq programmiert\grqq werden
	\end{itemize}
	\paragraph{Anwendung:}
	\begin{itemize}
		\item Anwendungsspezifische IC bei kleinen und mittleren Stückzahlen
		\item Mehrfach neu programmierbar zwecks Optimierung und Fehlerbehebung, auch während des praktischen Einsatzes
		\item Einfache Integration eines ganzen Systems auf einem Chip
		\item Prototyping, HW-/SW-Codesign
	\end{itemize}

\subsection{Hardwareprogrammierung}
	\begin{center}
	\includegraphics[page=21, width=0.7\linewidth, trim=30mm 20mm 20mm 40mm, clip]{\Path/resources/Vorlesung/VLSI/01_Einfuehrung.pdf}
	\end{center}
%\newpage
\subsection{Programmiertechnologien}
	\begin{center}
	\includegraphics[page=22, width=0.7\linewidth, trim=30mm 20mm 20mm 40mm, clip]{\Path/resources/Vorlesung/VLSI/01_Einfuehrung.pdf}
	\\\includegraphics[page=23, width=0.7\linewidth, trim=30mm 20mm 20mm 40mm, clip]{\Path/resources/Vorlesung/VLSI/01_Einfuehrung.pdf}
	\\\includegraphics[page=24, width=0.7\linewidth, trim=30mm 20mm 20mm 40mm, clip]{\Path/resources/Vorlesung/VLSI/01_Einfuehrung.pdf}
	\\\includegraphics[page=25, width=0.7\linewidth, trim=30mm 20mm 20mm 40mm, clip]{\Path/resources/Vorlesung/VLSI/01_Einfuehrung.pdf}
	\\\includegraphics[page=26, width=0.7\linewidth, trim=30mm 20mm 20mm 40mm, clip]{\Path/resources/Vorlesung/VLSI/01_Einfuehrung.pdf}
	\\\includegraphics[page=27, width=0.7\linewidth, trim=30mm 20mm 20mm 40mm, clip]{\Path/resources/Vorlesung/VLSI/01_Einfuehrung.pdf}
	\\\includegraphics[page=28, width=0.7\linewidth, trim=30mm 20mm 20mm 40mm, clip]{\Path/resources/Vorlesung/VLSI/01_Einfuehrung.pdf}
	\end{center}

\subsection{Klassifikation Anwenderprogrammierbare IC}
	\begin{center}
	\includegraphics[page=29, width=0.7\linewidth, trim=30mm 20mm 20mm 40mm, clip]{\Path/resources/Vorlesung/VLSI/01_Einfuehrung.pdf}
	\\\includegraphics[page=30, width=0.7\linewidth, trim=30mm 20mm 20mm 40mm, clip]{\Path/resources/Vorlesung/VLSI/01_Einfuehrung.pdf}
	\\\includegraphics[page=31, width=0.7\linewidth, trim=30mm 20mm 20mm 40mm, clip]{\Path/resources/Vorlesung/VLSI/01_Einfuehrung.pdf}
	\end{center}

\newpage




\part{Parallelverarbeitung}
\newpage
%\input{../Parallelrechner/Skript/00-Parallelrechner}

\part{Appendix}
\newpage
\chapter{VLSI-Systementwurf Praktikum}
	\section{Kurze Beschreibung des Terasic DE0 Board}
	\begin{center}
	\includegraphics[page=1, width=\linewidth, trim=30mm 30mm 20mm 110mm, clip]{\Path/resources/Praktikum/VLSI/Material/Praktikumsboard/DE0_Description.pdf}
	\begin{center}
	\end{center}
	\includegraphics[page=2, width=\linewidth, trim=30mm 90mm 20mm 30mm, clip]{\Path/resources/Praktikum/VLSI/Material/Praktikumsboard/DE0_Description.pdf}
	\end{center}
	
	\def \Path {/media/Daten/Studium/Skripte/TechnischeInformatik/resources/Praktikum/VLSI/Abgabe/Protokolle}
	\begin{center}
	\includegraphics[page=1, width=\linewidth, trim=30mm 100mm 20mm 110mm, clip]{\Path/resources/Stoppuhr.pdf}
	\end{center}
	\section{Aufgabe 1 - Code-Optimierung}

%\subsection{Aufgabenstellung}
%Für die Multiplikation zweier nxn-Matrizen soll ein möglichst effizienter Algorithmus
%gefunden werden. Nutzen Sie dazu den vorgegebenen Quelltext, der bereits die
%Basisvariante und eine Zeitmessroutine enthält. Diese Basisvariante sollen Sie
%optimieren – zunächst ohne zusätzliche Compiler-Flags.


\label{lst:matmul0-orig}
\lstinputlisting[caption={matmul0.c-Original}, language=C, linerange=47-59, firstnumber=47]{../matmul0-original.c}


\subsection{Optimierungen}
\begin{enumerate}
	\item Speichern öfters genutzter, zusammengesetzter Werte \\
		\code{
			i\_mult\_dim = i * dim; \\
				i\_mult\_dim\_add\_k = i\_mult\_dim + k; \\
				k\_mult\_dim = k * dim;
			}
				
	\item halten des Wertes einer Matrix für Multiplikation (loop-interchange)\\
		 \code{ A{[} i\_mult\_dim\_add\_k {]} }
	
\end{enumerate}

\label{lst:matmul0}
\lstinputlisting[caption={matmul0.c-optimiert}, language=C, linerange=46-63, firstnumber=46]{../matmul0.c}

	\begin{center}
	\includegraphics[page=1, width=\linewidth, trim=30mm 45mm 20mm 198mm, clip]{\Path/resources/Stoppuhr.pdf}
	\end{center}
	\section{Aufgabe 2 - Zeitmessungen}
(nicht Taurus)

\begin{table}[!htb]
\caption{Original, gcc}
\begin{minipage}{.5\linewidth}
\centering
\subsection{Original, gcc, keine flags}
\begin{tabular}{|l|r|r|}
	\hline
	\textsc{Dimension} & \textsc{Runtime} & \textsc{GFLOP/s} \\
	\hline
	\hline
	32  &  0.0002s  & 0.40 \\ 
	\hline 
	64  &  0.0015s  & 0.40 \\ 
	\hline 
	96  &  0.0045s  & 0.40 \\ 
	\hline 
	128  &  0.0115s  & 0.38 \\ 
	\hline 
	160  &  0.0217s  & 0.37 \\ 
	\hline 
	192  &  0.0503s  & 0.36 \\ 
	\hline 
	224  &  0.0651s  & 0.35 \\ 
	\hline 
	256  &  0.1365s  & 0.25 \\ 
	\hline 
	320  &  0.1984s  & 0.33 \\ 
	\hline 
	384  &  0.4676s  & 0.25 \\ 
	\hline 
	448  &  0.5567s  & 0.33 \\ 
	\hline 
	512  &  1.3026s  & 0.21 \\ 
	\hline 
	640  &  2.9606s  & 0.18 \\ 
	\hline 
	768  &  5.5037s  & 0.16 \\ 
	\hline 
	896  & 8.5051s  & 0.17 \\ 
	\hline 
	1024  & 26.0022s  & 0.16 \\ 
	\hline 

\end{tabular}
\end{minipage}%
\begin{minipage}{.5\linewidth}
\centering
\subsection{Original, gcc, -O3}
\begin{tabular}{|l|r|r|}
	\hline
	\textsc{Dimension} & \textsc{Runtime} & \textsc{GFLOP/s} \\
	\hline
	\hline
	32  &  0.0000s  & 1.73 \\ 
	\hline 
	64  &  0.0003s  & 1.85 \\ 
	\hline 
	96  &  0.0009s  & 1.93 \\ 
	\hline 
	128  &  0.0031s  & 1.35 \\ 
	\hline 
	160  &  0.0055s  & 1.56 \\ 
	\hline 
	192  &  0.0104s  & 1.37 \\ 
	\hline 
	224  &  0.0159s  & 1.42 \\ 
	\hline 
	256  &  0.0339s  & 1.00 \\ 
	\hline 
	320  &  0.0537s  & 1.24 \\ 
	\hline 
	384  &  0.1077s  & 1.06 \\ 
	\hline 
	448  &  0.1524s  & 1.18 \\ 
	\hline 
	512  &  0.3500s  & 0.78 \\ 
	\hline 
	640  &  1.6626s  & 0.33 \\ 
	\hline 
	768  &  3.2088s  & 0.29 \\ 
	\hline 
	896  & 5.1806s  & 0.28 \\ 
	\hline 
	1024  & 7.5068s  & 0.29 \\ 
	\hline 

\end{tabular}

\end{minipage}%
\end{table}


\begin{table}[!htb]
\caption{Mit Optimierungen, gcc}
\begin{minipage}{.5\linewidth}
\centering
\subsection{Mit Optimierungen, gcc, keine flags}
\begin{tabular}{|l|r|r|}
	\hline
	\textsc{Dimension} & \textsc{Runtime} & \textsc{GFLOP/s} \\
	\hline
	\hline
	32  &  0.0001s  & 0.61 \\ 
	\hline 
	64  &  0.0009s  & 0.61 \\ 
	\hline 
	96  &  0.0026s  & 0.70 \\ 
	\hline 
	128  &  0.0058s  & 0.71 \\ 
	\hline 
	160  &  0.0115s  & 0.71 \\ 
	\hline 
	192  &  0.0198s  & 0.71 \\ 
	\hline 
	224  &  0.0313s  & 0.72 \\ 
	\hline 
	256  &  0.0465s  & 0.72 \\ 
	\hline 
	320  &  0.0902s  & 0.73 \\ 
	\hline 
	384  &  0.1552s  & 0.73 \\ 
	\hline 
	448  &  0.2451s  & 0.73 \\ 
	\hline 
	512  &  0.3632s  & 0.74 \\ 
	\hline 
	640  &  0.7120s  & 0.74 \\ 
	\hline 
	768  &  1.2261s  & 0.74 \\ 
	\hline 
	896  &  1.9538s  & 0.74 \\ 
	\hline 
	1024  &  2.9417s  & 0.73 \\ 
	\hline 
\end{tabular}
\end{minipage}%
\begin{minipage}{.5\linewidth}
\centering
\subsection{Mit Optimierungen, gcc, -O3}
\begin{tabular}{|l|r|r|}
	\hline
	\textsc{Dimension} & \textsc{Runtime} & \textsc{GFLOP/s} \\
	\hline
	\hline
	32  &  0.0000s  & 2.12 \\ 
	\hline 
	64  &  0.0002s  & 2.24 \\ 
	\hline 
	96  &  0.0006s  & 2.50 \\ 
	\hline 
	128  &  0.0015s  & 2.58 \\ 
	\hline 
	160  &  0.0031s  & 2.56 \\ 
	\hline 
	192  &  0.0052s  & 2.65 \\ 
	\hline 
	224  &  0.0081s  & 2.67 \\ 
	\hline 
	256  &  0.0120s  & 2.75 \\ 
	\hline 
	320  &  0.0239s  & 2.72 \\ 
	\hline 
	384  &  0.0408s  & 2.76 \\ 
	\hline 
	448  &  0.0647s  & 2.77 \\ 
	\hline 
	512  &  0.0941s  & 2.86 \\ 
	\hline 
	640  &  0.1886s  & 2.78 \\ 
	\hline 
	768  &  0.3226s  & 2.81 \\ 
	\hline 
	896  &  0.5146s  & 2.80 \\ 
	\hline 
	1024  &  0.7530s  & 2.85 \\ 
	\hline 
\end{tabular}
\end{minipage}%
\end{table}


\begin{table}[!htb]
\caption{Original, icc}
\begin{minipage}{.5\linewidth}
\centering
\subsection{Original, icc, keine flags}
\begin{tabular}{|l|r|r|}
	\hline
	\textsc{Dimension} & \textsc{Runtime} & \textsc{GFLOP/s} \\
	\hline
	\hline
	32  &  0.0000s  & 1.52 \\ 
	\hline 
	64  &  0.0001s  & 3.67 \\ 
	\hline 
	96  &  0.0003s  & 4.30 \\ 
	\hline 
	128  &  0.0008s  & 4.88 \\ 
	\hline 
	160  &  0.0018s  & 4.87 \\ 
	\hline 
	192  &  0.0030s  & 4.90 \\ 
	\hline 
	224  &  0.0046s  & 4.73 \\ 
	\hline 
	256  &  0.0072s  & 4.73 \\ 
	\hline 
	320  &  0.0125s  & 4.99 \\ 
	\hline 
	384  &  0.0228s  & 5.02 \\ 
	\hline 
	448  &  0.0346s  & 5.17 \\ 
	\hline 
	512  &  0.0533s  & 5.02 \\ 
	\hline 
	640  &  0.1020s  & 5.13 \\ 
	\hline 
	768  &  0.1739s  & 5.15 \\ 
	\hline 
	896  &  0.2743s  & 5.25 \\ 
	\hline 
	1024  &  0.4137s  & 5.19 \\ 
	\hline 
\end{tabular}
\end{minipage}%
\begin{minipage}{.5\linewidth}
\centering
\subsection{Original, icc, -O3}
\begin{tabular}{|l|r|r|}
	\hline
	\textsc{Dimension} & \textsc{Runtime} & \textsc{GFLOP/s} \\
	\hline
	\hline
	32  &  0.0000s  & 1.98 \\ 
	\hline 
	64  &  0.0001s  & 3.69 \\ 
	\hline 
	96  &  0.0004s  & 4.00 \\ 
	\hline 
	128  &  0.0009s  & 4.51 \\ 
	\hline 
	160  &  0.0016s  & 4.67 \\ 
	\hline 
	192  &  0.0031s  & 4.60 \\ 
	\hline 
	224  &  0.0051s  & 4.43 \\ 
	\hline 
	256  &  0.0076s  & 4.40 \\ 
	\hline 
	320  &  0.0144s  & 4.57 \\ 
	\hline 
	384  &  0.0245s  & 4.66 \\ 
	\hline 
	448  &  0.0386s  & 4.67 \\ 
	\hline 
	512  &  0.0582s  & 4.60 \\ 
	\hline 
	640  &  0.1117s  & 4.70 \\ 
	\hline 
	768  &  0.1937s  & 4.69 \\ 
	\hline 
	896  &  0.3035s  & 4.75 \\ 
	\hline 
	1024  &  0.4552s  & 4.73 \\ 
	\hline 
\end{tabular}
\end{minipage}%
\end{table}




\begin{table}[!htb]
\caption{Mit Optimierungen, icc}
\begin{minipage}{.5\linewidth}
\centering
\subsection{Mit Optimierungen, icc, keine flags}
\begin{tabular}{|l|r|r|}
	\hline
	\textsc{Dimension} & \textsc{Runtime} & \textsc{GFLOP/s} \\
	\hline
	\hline
	32  &  0.0000s  & 2.99 \\ 
	\hline 
	64  &  0.0001s  & 5.09 \\ 
	\hline 
	96  &  0.0003s  & 5.41 \\ 
	\hline 
	128  &  0.0007s  & 5.98 \\ 
	\hline 
	160  &  0.0014s  & 5.95 \\ 
	\hline 
	192  &  0.0024s  & 5.65 \\ 
	\hline 
	224  &  0.0041s  & 5.41 \\ 
	\hline 
	256  &  0.0062s  & 5.29 \\ 
	\hline 
	320  &  0.0121s  & 5.45 \\ 
	\hline 
	384  &  0.0202s  & 5.52 \\ 
	\hline 
	448  &  0.0327s  & 5.57 \\ 
	\hline 
	512  &  0.0481s  & 5.61 \\ 
	\hline 
	640  &  0.0928s  & 5.63 \\ 
	\hline 
	768  &  0.1606s  & 5.65 \\ 
	\hline 
	896  &  0.2522s  & 5.70 \\ 
	\hline 
	1024  &  0.3757s  & 5.72 \\ 
	\hline 
\end{tabular}
\end{minipage}%
\begin{minipage}{.5\linewidth}
\centering
\subsection{Mit Optimierungen, icc, -O3}
\begin{tabular}{|l|r|r|}
	\hline
	\textsc{Dimension} & \textsc{Runtime} & \textsc{GFLOP/s} \\
	\hline
	\hline
	32  &  0.0000s  & 2.72 \\ 
	\hline 
	64  &  0.0001s  & 5.07 \\ 
	\hline 
	96  &  0.0003s  & 5.34 \\ 
	\hline 
	128  &  0.0006s  & 5.67 \\ 
	\hline 
	160  &  0.0016s  & 5.60 \\ 
	\hline 
	192  &  0.0024s  & 5.78 \\ 
	\hline 
	224  &  0.0046s  & 5.35 \\ 
	\hline 
	256  &  0.0062s  & 5.32 \\ 
	\hline 
	320  &  0.0122s  & 5.44 \\ 
	\hline 
	384  &  0.0202s  & 5.51 \\ 
	\hline 
	448  &  0.0322s  & 5.58 \\ 
	\hline 
	512  &  0.0482s  & 5.57 \\ 
	\hline 
	640  &  0.0950s  & 5.52 \\ 
	\hline 
	768  &  0.1650s  & 5.50 \\ 
	\hline 
	896  &  0.2561s  & 5.54 \\ 
	\hline 
	1024  &  0.3841s  & 5.58 \\ 
	\hline 
\end{tabular}
\end{minipage}%
\end{table}

	\begin{center}
	\includegraphics[page=2, width=\linewidth, trim=30mm 130mm 20mm 30mm, clip]{\Path/resources/Stoppuhr.pdf}
	\end{center}
	\section{Aufgabe 3 - Compiler-Flags}

\subsection{-O3}
Bei Verwendung des gcc-Compilers bringt dieser Flag eine Verbesserung der Ausführungszeit vom Faktor 4 mit sich. Allerdings verschlechtert er die Ausführungszeit beim icc.
\subsection{-floop-interchange}
Führt eine Vertauschung von Schleifen aus, ähnlich wie die Code-Optimierung.
\subsection{-funroll-loops}
Nimmt Schleifen auseinander, deren Schritte durch den Compiler vor der Ausführung bestimmt werden können.



	\begin{center}
	\includegraphics[page=3, width=\linewidth, trim=30mm 100mm 20mm 30mm, clip]{\Path/resources/Stoppuhr.pdf}
	\end{center}
	\section{Aufgabe 4 - FLOP/s}
Intel E5-2690 Sandy Bridge \\\\
Maximaler (Daten-)Cache per Core: 20MB L3 + 256KB L2 + 32KB L1 = 20,35 MB \\\\
FLOP/s je Kern (Normal Clock): $2,9GHz\;*\;\frac{8\; FLOPs}{cycle}\; = \;23,2 \;GFLOP/s\; per\; Core$ \\\\
Notwendige Speicherbandbreite (für einen Kern): $23,2 \;GFLOP/s * 32 \frac{Bit}{FLOP} \;=\; 92,8 \;GB/s$ \\\\
FLOP/s je Kern (Maximum Clock): $3,8GHz\;*\;\frac{8\; FLOPs}{cycle}\; = \;30,4 \;GFLOP/s \; per \; Core$ \\\\
Notwendige Speicherbandbreite (für einen Kern): $30,4 \;GFLOP/s * 32 \frac{Bit}{FLOP} \;=\; 121,6\; GB/s$ \\\\
L1-Cache: 32KB Daten, 4 Takte Latenz, 
Tatsächliche Speicherbandbreite (ein Kanal, DDR3-1600): $1600MT/s\; *\; 8 Bit\; = \;12,8 \;GB/s$
\\\\
Verbrauchter Speicher (3 Matritzen mit jeweils 1 Mio Einträgen): $3 * 1000000 * 32 = 12MB$ \\\\
%Obwohl der Prozessor eine weitaus höhere Anzahl an Floating Point-Operationen pro Sekunde zulassen würde, hat der Speicher-Bus nicht ausreichend MT/s. 
Dass maximal in etwa ein Viertel (ca. 6 GFLOP/s) der möglichen 23,2 GFLOP/s erreicht wird, liegt an der Cache-Latenz. Da der benötigte Speicherplatz die Kapazität der Caches nicht überschreitet, müssen keine Daten aus dem RAM geladen werden. \\
Jedoch können pro Takt nur 4 Floats aus dem L1,2 Floats aus L2 und 1 Float aus L3 geladen werden, was maximal 7 Floats per Takt liefert (bei einer idealen Verteilung). Um tatsächlich alle FPUs des Kerns auszulasten, müssten pro Operand mehrere Operationen ausgeführt werden.
\\\\


	\begin{center}
	\includegraphics[page=4, width=\linewidth, trim=30mm 130mm 20mm 30mm, clip]{\Path/resources/Stoppuhr.pdf}
	\end{center}
	\input{\Path/05-Aufgabe}
	\begin{center}
	\includegraphics[page=4, width=\linewidth, trim=30mm 25mm 20mm 175mm, clip]{\Path/resources/Stoppuhr.pdf}
	\includegraphics[page=5, width=\linewidth, trim=30mm 200mm 20mm 30mm, clip]{\Path/resources/Stoppuhr.pdf}
	\end{center}
	\input{\Path/06-Aufgabe}
	\chapter{Anhang}
\section{Circuit}
\label{lst:circuit}
\lstinputlisting[caption={Circuit.h}, language=C++]{../Simulator/src/main/Circuit.h}

\section{Simulator}
\label{lst:simulator}
\lstinputlisting[caption={Simulator.h}, language=C++]{../Simulator/src/main/Simulator.h}

\section{Solver}
\label{lst:solver}
\lstinputlisting[caption={Solver.h}, language=C++]{../Simulator/src/main/Solver.h}

\section{Parsers}
\label{lst:parsers}
\lstinputlisting[caption={Parsers.h}, language=C++]{../Simulator/src/parser/Parsers.h}

\section{Parser}
\label{lst:parser}
\lstinputlisting[caption={Parser.h}, language=C++]{../Simulator/src/parser/Parser.h}

\section{BENCH}
\label{lst:bench}
\lstinputlisting[caption={BENCH.h}, language=C++]{../Simulator/src/parser/BENCH.h}

\section{Gates}
\label{lst:gates}
\lstinputlisting[caption={Gates.h}, language=C++]{../Simulator/src/gates/Gates.h}

\section{Gate}
\label{lst:gate}
\lstinputlisting[caption={Gate.h}, language=C++]{../Simulator/src/gates/Gate.h}

\section{Input}
\label{lst:input}
\lstinputlisting[caption={Input.h}, language=C++]{../Simulator/src/gates/Input.h}

\section{DFF}
\label{lst:dff}
\lstinputlisting[caption={DFF.h}, language=C++]{../Simulator/src/gates/DFF.h}

\section{Output}
\label{lst:output}
\lstinputlisting[caption={Output.h}, language=C++]{../Simulator/src/gates/Output.h}
\newpage


	\def \Path {/media/Daten/Studium/Skripte/TechnischeInformatik}


\chapter{Entwurf eingebetteter Systeme Praktikum}
	

\end{document}
