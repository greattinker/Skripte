\chapter{Einführung}
\section{Themenschwerpunkte}
	\begin{description}

		\item[Verarbeitungsleistung]\hfill
			\\\textbf{Im Vordergrund stehen:}
			\begin{itemize}
				\item Schnelle Verarbeitung auch einzelner Bits
				\item Parallelität auf: \hfill
				\begin{itemize}
					\item Bitebene
					\item Befehlseben
					\item Threadebene
					\item Prozess- und Anwendungsebene
				\end{itemize}
				\item dynamische Rekonfiguration
			\end{itemize}
			$\Rightarrow$ Erfüllung der gegebenen Anforderungen

		\item[Systemintegration]\hfill
			\\\textbf{Im Vordergrund stehen:}
			\begin{itemize}
				\item Mehrprozessorsyteme, Mehrkern-, Vielkernprozessorsysteme
				\item Mehr-Chip- / Einzel-Chip-Lösungen (System-on-a-Chip)
				\item Parallele Entwicklung von HW und SW
	(HW-/SW-Codesign)
				\item System-Prototyping (FPGA-Entwurf)
			\end{itemize}
			$\Rightarrow$ Kosteneinsparung, Entwicklungszeiteinsparung (Time-to-Market)

		\item[Verlustleistung]
			\hfill\\\textbf{Im Vordergrund stehen:}
			\begin{itemize}
				\item Verlustleistung im Standby (Akkubetrieb)
				\item Maximales Abwärmebudget
			\end{itemize}

			\hfill\\\textbf{Kenngrößen:}
			\begin{itemize}
				\item Statische und dynamische Verlustleistung
				\item MIPS pro Watt
			\end{itemize}

			$\Rightarrow$ Sowohl für eingebettete Systeme als auch für Server


		\item[Korrektheit]
			\hfill\\\textbf{Aspekte:}
			\begin{itemize}
				\item Verifikation eines Schaltkreises, simulativ / formal
				\item Profiling und Debugging unter Echtzeitbedingungen (Trace)
			\end{itemize}
			$\Rightarrow$ Fehlerfreier Erstentwurf


		\item[Fehlertoleranz]\hfill
			\\\textbf{Toleranz gegenüber:}
			\begin{itemize}
				\item Permanenten Fehlern (zeitunabhängig nach erstem Auftreten)
				\item Intermitierenden Fehlern (nur unter bestimmten Betriebsbedingungen)
				\item Transienten Fehlern (aufgrund statischer Störungen)
			\end{itemize}

			\hfill\\\textbf{Fehlererkennung und -korrektur:}
			\begin{itemize}
				\item Autonom durch Hardwarearchitektur
				\item Aus Kombination von HW und SW
			\end{itemize}

			$\Rightarrow$ Insbesondere wichtig für Sicherheitskritische, hochverfügbare und langlebige zuverlässige Systeme
			\\$\Rightarrow$ Steigende Siginifikanz mit abnehmenden Strukturgrößen (Integrationsgrad) sowie steigender Transistoranzahl (Moore's Law)
	\end{description}


\section{Inhalte der Lehrveranstaltung}
\subsection{Inhalte der Vorlesung}
	\begin{enumerate}
		\item Klassifikation von Schaltkreisen
		\item Grundlagen des Schaltkreisentwurfs
		\item Automatendarstellung, -kopplung, -vereinfachung
		\item Hardwarebeschreibungssprachen
		\item Programmierabare Schaltkreise, insbesondere FPGAs - Teil 1
		\item Programmierabare Schaltkreise, insbesondere FPGAs - Teil 2
		\item Modellierung und Simulation
		\item Zeitverhalten und Test
		\item Hochgeschwindigkeit und Verlsustleistung
		\item Anwendungsbeispiele
	\end{enumerate}

\subsection{Inhalte des Praktikums}
	\begin{enumerate}
		\item Altera Quartus-Toolchain \& Praktikumsboard DE0 mit Cyclone-3
		\item Schaltnetze und Schaltwerke
		\item Modularisierung
		\item \grqq Komplexe\grqq Anwendung: Stoppuhr
	\end{enumerate}


\section{Klassifikation von ICs}

\subsection{Zwei Sichten}
	\hfill\\Klassifizierung von integrierten Schaltkreisen (ICs) in
	\begin{itemize}
		\item Standarschaltkreis (Standard-IC) und
		\item applikationspezifische Schaltkreise (Application-Specific IC, \textbf{ASIC})
	\end{itemize}
	unter zwei Gesichtspunkten möglich:
	\begin{itemize}
		\item \textbf{Herstellungssicht}
		\item \textbf{Entwurfssicht}
	\end{itemize}

	\paragraph{Herstellungssicht:}
	\begin{itemize}
		\item Standard-IC = große Stückzahl für viele Kunden
		\item ASIC = für eine/n Kunden/Applikation speziell entwickelter und gefertigter IC mit zugeschnittener Funktionalität
	\end{itemize}

	\paragraph{Entwurfssicht:}
	\begin{itemize}
		\item Standard-IC = (Hardware-)Funktionalität kann nicht vom Anwender beeinflusst werden
		\item ASIC = vom Anwender selbst entwickelte (Hardware-)Funktionalität
	\end{itemize}

	\newpage
	\begin{center}
	\includegraphics[page=17, width=0.7\linewidth, trim=30mm 20mm 20mm 40mm, clip]{\Path/resources/Vorlesung/VLSI/01_Einfuehrung.pdf}
	\\\includegraphics[page=18, width=0.7\linewidth, trim=30mm 20mm 20mm 40mm, clip]{\Path/resources/Vorlesung/VLSI/01_Einfuehrung.pdf}
	\\\includegraphics[page=19, width=0.7\linewidth, trim=30mm 20mm 20mm 40mm, clip]{\Path/resources/Vorlesung/VLSI/01_Einfuehrung.pdf}
	\end{center}

\subsection{Anwenderprogrammierbare IC (ASIC)}
	\paragraph{Merkmale:}
	\begin{itemize}
		\item Field-Programmable $\Leftrightarrow$ feldprogrammierbar
		\item Vor Ort (im Feld) vom Anwender programmierbar
		\item Hardware ist fix. Funktionalität kann aber mittels spezieller Konfiguration \grqq programmiert\grqq werden
	\end{itemize}
	\paragraph{Anwendung:}
	\begin{itemize}
		\item Anwendungsspezifische IC bei kleinen und mittleren Stückzahlen
		\item Mehrfach neu programmierbar zwecks Optimierung und Fehlerbehebung, auch während des praktischen Einsatzes
		\item Einfache Integration eines ganzen Systems auf einem Chip
		\item Prototyping, HW-/SW-Codesign
	\end{itemize}

\subsection{Hardwareprogrammierung}
	\begin{center}
	\includegraphics[page=21, width=0.7\linewidth, trim=30mm 20mm 20mm 40mm, clip]{\Path/resources/Vorlesung/VLSI/01_Einfuehrung.pdf}
	\end{center}
%\newpage
\subsection{Programmiertechnologien}
	\begin{center}
	\includegraphics[page=22, width=0.7\linewidth, trim=30mm 20mm 20mm 40mm, clip]{\Path/resources/Vorlesung/VLSI/01_Einfuehrung.pdf}
	\\\includegraphics[page=23, width=0.7\linewidth, trim=30mm 20mm 20mm 40mm, clip]{\Path/resources/Vorlesung/VLSI/01_Einfuehrung.pdf}
	\\\includegraphics[page=24, width=0.7\linewidth, trim=30mm 20mm 20mm 40mm, clip]{\Path/resources/Vorlesung/VLSI/01_Einfuehrung.pdf}
	\\\includegraphics[page=25, width=0.7\linewidth, trim=30mm 20mm 20mm 40mm, clip]{\Path/resources/Vorlesung/VLSI/01_Einfuehrung.pdf}
	\\\includegraphics[page=26, width=0.7\linewidth, trim=30mm 20mm 20mm 40mm, clip]{\Path/resources/Vorlesung/VLSI/01_Einfuehrung.pdf}
	\\\includegraphics[page=27, width=0.7\linewidth, trim=30mm 20mm 20mm 40mm, clip]{\Path/resources/Vorlesung/VLSI/01_Einfuehrung.pdf}
	\\\includegraphics[page=28, width=0.7\linewidth, trim=30mm 20mm 20mm 40mm, clip]{\Path/resources/Vorlesung/VLSI/01_Einfuehrung.pdf}
	\end{center}

\subsection{Klassifikation Anwenderprogrammierbare IC}
	\begin{center}
	\includegraphics[page=29, width=0.7\linewidth, trim=30mm 20mm 20mm 40mm, clip]{\Path/resources/Vorlesung/VLSI/01_Einfuehrung.pdf}
	\\\includegraphics[page=30, width=0.7\linewidth, trim=30mm 20mm 20mm 40mm, clip]{\Path/resources/Vorlesung/VLSI/01_Einfuehrung.pdf}
	\\\includegraphics[page=31, width=0.7\linewidth, trim=30mm 20mm 20mm 40mm, clip]{\Path/resources/Vorlesung/VLSI/01_Einfuehrung.pdf}
	\end{center}

\newpage
