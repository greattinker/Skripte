\chapter{Automaten}

\section{Automatendarstellung}

\subsection{Betrachtungsweisen}
	\paragraph{Verschiedene Sichten/Semantik:}
	\begin{itemize}
		\item Zustandsübergangsdiagramm (state diagram)
			\begin{itemize}
				\item Vernetzung von Zuständen
				\item Beispiele: UML-Zustandsdiagramm, Automatengrpahen, SM Charts, GRAFCET, Sequential Function Charts (SFC)
			\end{itemize}
		\item Ablaufdiagramm (flow chart)
			\begin{itemize}
				\item Vernetzung von Prozessen
				\item Zustand ergibt sich aus der Verkettung aller Variablenzustände
				\item Beispiele: UML-Aktivitätsdiagramm, Programmablaufplan (PAP)
			\end{itemize}
	\end{itemize}
	
	\subsubsection{Binärzähler mod 4 mit Trigger X und Übertrag Y}	
		\begin{center}
			\includegraphics[page=4, width=0.6\linewidth, trim=35mm 27mm 30mm 58mm, clip]{\Path/resources/Vorlesung/VLSI/03_Automaten.pdf}
		\end{center}
	
\subsection{Automatengraphen}	
	\begin{center}
		\includegraphics[page=5, width=0.8\linewidth, trim=35mm 67mm 30mm 58mm, clip]{\Path/resources/Vorlesung/VLSI/03_Automaten.pdf}
	\end{center}
	\paragraph{Endlicher Automat:} Menge der möglichen Eingabezeichen, Ausgabezeichen und inneren Zustände ist endlich ...
	\paragraph{Synchron getakteter Automat:} Zustandsübergänge aller Speicherglieder erfolgen gleichzeitig, synchron zu einem Taktsignal
	
	\subsubsection{Notation}
		\begin{center}
			\includegraphics[page=6, width=0.65\linewidth, trim=35mm 37mm 30mm 58mm, clip]{\Path/resources/Vorlesung/VLSI/03_Automaten.pdf}
		\end{center}
	\subsubsection{Grafische Darstellung}
		\begin{center}
			\includegraphics[page=7, width=0.65\linewidth, trim=35mm 39mm 30mm 58mm, clip]{\Path/resources/Vorlesung/VLSI/03_Automaten.pdf}
		\end{center}
		\paragraph{$\Rightarrow$ Prüfung auf:} Vollständigkeit und Widerspruchfreiheit
	
	\subsubsection{Getaktete Automaten}
		\paragraph{Theoretische Informatik}
		\begin{itemize}
			\item Verarbeitung von EIngabezeichen sofern vorhanden
			\item Jedes Zeichen in der Eingabe wird einmalig verarbeitet
		\end{itemize} 
		\paragraph{Technische Informatik}\hfill\\
		Taktung des Automaten mit einem Taktsignal $\rightarrow$ Abtatstung der Eingabe\\
		Konsequenzen:
		\begin{itemize}
			\item Zeichen für \grqq keine Eingabe \grqq erforderlich
			\item Abtastung \grqq derselben Eingabe \grqq in aufeinanderfolgenden Takten möglich
		\end{itemize} 
		$\Rightarrow$ Korrekte Modellierung des Taktsignals erforderlich
	
		\paragraph{Beispiel - Taktzustandsgesteuertes D-Flip-Flop (Latch)}\hfill\\
		(Taktsignal C als Eingabe) 
		\begin{center}
			\includegraphics[page=9, width=0.65\linewidth, trim=35mm 50mm 30mm 80mm, clip]{\Path/resources/Vorlesung/VLSI/03_Automaten.pdf}
		\end{center}
	
		\paragraph{Beispiel - Taktflankengesteuertes D-Flip-Flop}\hfill\\
		(Taktsignal C als Eingabe) 
		\begin{center}
			\includegraphics[page=10, width=0.65\linewidth, trim=35mm 26mm 30mm 80mm, clip]{\Path/resources/Vorlesung/VLSI/03_Automaten.pdf}
		\end{center}
	
		\paragraph{Beispiel - Taktflankengesteuertes D-Flip-Flop}\hfill\\
		(Definition: Zustandsübergang nur bei taktflanke) 
		\begin{center}
			\includegraphics[page=11, width=0.65\linewidth, trim=35mm 50mm 30mm 80mm, clip]{\Path/resources/Vorlesung/VLSI/03_Automaten.pdf}
		\end{center}
	
\subsection{SM-Charts}
		\begin{center}
			\includegraphics[page=12, width=0.65\linewidth, trim=35mm 30mm 30mm 60mm, clip]{\Path/resources/Vorlesung/VLSI/03_Automaten.pdf}
		\end{center}
		
		\subsubsection{Verzweigungen}
			\paragraph{Weitere Merkmale:}
			\begin{itemize}
				\item Selbstdefinierte Abkürzungen für komplexe Ausdrücke üblich
			\end{itemize}
	
