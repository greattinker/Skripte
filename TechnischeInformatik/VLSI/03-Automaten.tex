\chapter{Automaten}

\section{Automatendarstellung}

\subsection{Betrachtungsweisen}
	\paragraph{Verschiedene Sichten/Semantik:}
	\begin{itemize}
		\item Zustandsübergangsdiagramm (state diagram)
			\begin{itemize}
				\item Vernetzung von Zuständen
				\item Beispiele: UML-Zustandsdiagramm, Automatengrpahen, SM Charts, GRAFCET, Sequential Function Charts (SFC)
			\end{itemize}
		\item Ablaufdiagramm (flow chart)
			\begin{itemize}
				\item Vernetzung von Prozessen
				\item Zustand ergibt sich aus der Verkettung aller Variablenzustände
				\item Beispiele: UML-Aktivitätsdiagramm, Programmablaufplan (PAP)
			\end{itemize}
	\end{itemize}
	
	\subsubsection{Binärzähler mod 4 mit Trigger X und Übertrag Y}	
		\begin{center}
			\includegraphics[page=4, width=0.6\linewidth, trim=35mm 27mm 30mm 58mm, clip]{\Path/resources/Vorlesung/VLSI/03_Automaten.pdf}
		\end{center}
	
\subsection{Automatengraphen}	
	\begin{center}
		\includegraphics[page=5, width=0.8\linewidth, trim=35mm 67mm 30mm 58mm, clip]{\Path/resources/Vorlesung/VLSI/03_Automaten.pdf}
	\end{center}
	\paragraph{Endlicher Automat:} Menge der möglichen Eingabezeichen, Ausgabezeichen und inneren Zustände ist endlich ...
	\paragraph{Synchron getakteter Automat:} Zustandsübergänge aller Speicherglieder erfolgen gleichzeitig, synchron zu einem Taktsignal
	
	\subsubsection{Notation}
		\begin{center}
			\includegraphics[page=6, width=0.65\linewidth, trim=35mm 37mm 30mm 58mm, clip]{\Path/resources/Vorlesung/VLSI/03_Automaten.pdf}
		\end{center}
	\subsubsection{Grafische Darstellung}
		\begin{center}
			\includegraphics[page=7, width=0.65\linewidth, trim=35mm 39mm 30mm 58mm, clip]{\Path/resources/Vorlesung/VLSI/03_Automaten.pdf}
		\end{center}
		\paragraph{$\Rightarrow$ Prüfung auf:} Vollständigkeit und Widerspruchfreiheit
	
	\subsubsection{Getaktete Automaten}
		\paragraph{Theoretische Informatik}
		\begin{itemize}
			\item Verarbeitung von EIngabezeichen sofern vorhanden
			\item Jedes Zeichen in der Eingabe wird einmalig verarbeitet
		\end{itemize} 
		\paragraph{Technische Informatik}\hfill\\
		Taktung des Automaten mit einem Taktsignal $\rightarrow$ Abtatstung der Eingabe\\
		Konsequenzen:
		\begin{itemize}
			\item Zeichen für \grqq keine Eingabe \grqq erforderlich
			\item Abtastung \grqq derselben Eingabe \grqq in aufeinanderfolgenden Takten möglich
		\end{itemize} 
		$\Rightarrow$ Korrekte Modellierung des Taktsignals erforderlich
	
		\paragraph{Beispiel - Taktzustandsgesteuertes D-Flip-Flop (Latch)}\hfill\\
		(Taktsignal C als Eingabe) 
		\begin{center}
			\includegraphics[page=9, width=0.65\linewidth, trim=35mm 50mm 30mm 80mm, clip]{\Path/resources/Vorlesung/VLSI/03_Automaten.pdf}
		\end{center}
	
		\paragraph{Beispiel - Taktflankengesteuertes D-Flip-Flop}\hfill\\
		(Taktsignal C als Eingabe) 
		\begin{center}
			\includegraphics[page=10, width=0.65\linewidth, trim=35mm 26mm 30mm 80mm, clip]{\Path/resources/Vorlesung/VLSI/03_Automaten.pdf}
		\end{center}
	
		\paragraph{Beispiel - Taktflankengesteuertes D-Flip-Flop}\hfill\\
		(Definition: Zustandsübergang nur bei taktflanke) 
		\begin{center}
			\includegraphics[page=11, width=0.65\linewidth, trim=35mm 50mm 30mm 80mm, clip]{\Path/resources/Vorlesung/VLSI/03_Automaten.pdf}
		\end{center}
	
\subsection{SM-Charts}
	\begin{center}
		\includegraphics[page=12, width=0.65\linewidth, trim=35mm 30mm 30mm 60mm, clip]{\Path/resources/Vorlesung/VLSI/03_Automaten.pdf}
	\end{center}
	
	\subsubsection{Verzweigungen}
		\paragraph{Weitere Merkmale:}
		\begin{itemize}
			\item Selbstdefinierte Abkürzungen für komplexe Ausdrücke üblich
			\item im Allgemeinen nur Prüfung einer Eingangsvariablen (nicht Eingabezeichen) pro Verzweigung
		\end{itemize}
		\paragraph{Weitere Beispiele:}
		\begin{center}
			\includegraphics[page=13, width=0.65\linewidth, trim=35mm 27mm 30mm 111mm, clip]{\Path/resources/Vorlesung/VLSI/03_Automaten.pdf}
		\end{center}
		
	
	\subsubsection{Ausgaben}
		\paragraph{Ebenso:}
		Selbstdefinierte Abkürzungen für komplexe Ausdrücke üblich
		
		\paragraph{Weitere Beispiele:}
		\begin{center}
			\includegraphics[page=14, width=0.65\linewidth, trim=35mm 27mm 30mm 110mm, clip]{\Path/resources/Vorlesung/VLSI/03_Automaten.pdf}
		\end{center}
	
	\subsubsection{Vorteile gegenüber Automatengraphen}
		\begin{itemize}
			\item prinzipiell Vollständig
			\item präzise Modellierung hierarchischer Verzweigungen
		\end{itemize}
		\begin{center}
			\includegraphics[page=15, width=0.65\linewidth, trim=35mm 27mm 30mm 80mm, clip]{\Path/resources/Vorlesung/VLSI/03_Automaten.pdf}
		\end{center}
			

\subsection{GRAFCET \& SFC}
	\paragraph{Allgemeines:}
	\begin{itemize}
		\item Struktur entspricht einem Petri-Netz
		\item Darstellung von schrittweise ausgeführten Ablaufbeschreibungen
		\item Anwendung in Automatisierungstechnik und Verfahrenstechnik
		\item Verwendbar zur Programmierung von Speicherprogrammierbaren Steuerungen (SPS)
		\item Automatenkopplung nicht vorgesehen; muss durch Eingangs- und Ausgangsvariablen realisiert werden
	\end{itemize}
	\hfill\\
	\begin{tabular}{ll}
		GRAFCET & GRAphe Functionnel de Commande Etapes/Transitions\\
		SFC & sequential function chart
	\end{tabular}
	
	
	\subsubsection{Grundelemente}
		\begin{itemize}
			\item Schritt:
				\begin{itemize}
					\item entspricht einem Zustand
					\item mindestens ein Initialisierungsschritt notwendig
				\end{itemize}
			\item Transition
				\begin{itemize}
					\item Schaltbedingung für Übergang zwischen zwei Schritten (boolesche Gleichung) = zeitliche Ereignisse
					\item Schritte und Transitionen folgen immer aufeinander
					\item Leserichtung: Transitionen können nur von oben nach unten durchlaufen werden
				\end{itemize}
			\item Aktion
				\begin{itemize}
					\item einem Schritt zugeordnet
					\item realisiert die Ausgabe
					\item verschiedene Aktionsarten möglich
				\end{itemize}
		\end{itemize}
	
	\subsubsection{Aktionen}
		\begin{center}
			\includegraphics[page=18, width=0.65\linewidth, trim=85mm 27mm 50mm 43mm, clip]{\Path/resources/Vorlesung/VLSI/03_Automaten.pdf}
		\end{center}
		
	\subsubsection{Kontrollfluss: Verzweigungen}
		\begin{center}
			\includegraphics[page=19, width=0.65\linewidth, trim=55mm 27mm 50mm 60mm, clip]{\Path/resources/Vorlesung/VLSI/03_Automaten.pdf}
		\end{center}
		
	\subsubsection{Kontrollfluss: Schleife}
		\begin{center}
			\includegraphics[page=20, width=0.65\linewidth, trim=45mm 27mm 50mm 60mm, clip]{\Path/resources/Vorlesung/VLSI/03_Automaten.pdf}
		\end{center}
		
	\subsubsection{Vergleich von GRAFCET \& SFC}
		\begin{itemize}
			\item identische Grundstruktur, Verzweigung und Initialisierung
			\item Ablauf-Ends in GRAFCET nicht notwendig, dann SPrung zu Initialisierungsschritt
			\item Schleifen in SFC
				\begin{itemize}
					\item Vorwärtssprung = Alternativverzweigung
					\item Rückwärtssprung = Schleife
				\end{itemize}
			\item Aktionen s.o.
		\end{itemize}
		$\Rightarrow$ GRAFCET ist für den Entwurf konzipiert, SFC für die Implementierung
		
	\subsubsection{Beispiel zu GRAFCET}
		\begin{center}
			\includegraphics[page=22, width=0.65\linewidth, trim=100mm 27mm 50mm 42mm, clip]{\Path/resources/Vorlesung/VLSI/03_Automaten.pdf}
		\end{center}
		


\section{Automatenkopplung}
	\paragraph{Zerlegung der Gesamtaufgabe in Teilschaltungen zwecks:}
	\begin{itemize}
		\item Nebenläufigkeit\\(statt sequentieller Abarbeitung)
		\item Energieeinsparung ducrh Abschaltung ungenutzter Komponenten\\(statt Dauerbetrieb der gesamten Schaltung)
		\item komponentenbasierter, testfreundlicher Entwurf\\(statt monolitischem Design)
		\item Integration von Komponenten von Drittanbietern\\(statt eigenem Entwurf aller Komponenten)
	\end{itemize}
	
	\paragraph{Außerdem:} 
	Kommunikation mit der Außenwelt erfordert im Allgemeinen nebenläufig arbeitende I/O-Controller aufgrund asynchron eintreffender Ereignisse (Nachrichten) $\rightarrow$ Client-Server Architektur
	
	\subsubsection{Zeitliche Kopplung}
		\begin{center}
			\includegraphics[page=25, width=0.65\linewidth, trim=30mm 50mm 30mm 65mm, clip]{\Path/resources/Vorlesung/VLSI/03_Automaten.pdf}
		\end{center}
		
	\subsubsection{Asynchrone Kopplung synchroner Teilschaltungen}
		\paragraph{Merkmal:} Mehrere (lokale) Taktsignale / Taktdomänen (clock domain)
		\paragraph{Vorteile:}
		\begin{itemize}
			\item Timing-Analyse sichert Zeitverhalten innerhalb der Domäne
			\item passende Taktfrequenz innerhalb der Domäne
			\begin{itemize}
				\item Kombination High-Speed und Low-Power
				\item Taktfrequenz dynamisch anpassbar
				\item Abschalten einer Domäne für Standby
			\end{itemize}
		\end{itemize}
		\paragraph{Nachteil:}\hfill\\
		Datenaustausch zwischen Taktdomänen erfordern:
		\begin{itemize}
			\item Module mit Single-Bit-Synchronizer, Cross-Clock-FIFOs
			\item spezielle Timing-Constraints
		\end{itemize}
	
	\subsubsection{Synchrone Kopplung mit mehreren Taktsignalen}
		\paragraph{Merkmal:} Mehrere (lokale) Taktsignale / Taktdomänen, zwischen denen aber spezielle Abhängigkeiten bestehen (dependent clocks)
		\paragraph{Vorteile:}
		\begin{itemize}
			\item Timing-Analyse sichert Zeitverhalten innerhalb der Domäne und auch zwischen den Domänen
			\item passende feste Taktfrequenz je Domäne
			\item keine speziellen Synchronisationselemente erforderlich
		\end{itemize}
		\paragraph{Nachteil:} Taktfrequenzen sind statisch.
		
		
\subsection{Synchrone Kopplung}
	Kommunikation über:
	\begin{itemize}
		\item Zustände oder
		\item Ausgaben
	\end{itemize}
	
	Anordnung:
	\begin{itemize}
		\item parallel oder
		\item seriell
	\end{itemize}
	
	
	\subsubsection{Kommunikation über Zustände}
		\begin{center}
			\includegraphics[page=30, width=0.65\linewidth, trim=30mm 27mm 30mm 60mm, clip]{\Path/resources/Vorlesung/VLSI/03_Automaten.pdf}
		\end{center}
	
	\subsubsection{Kommunikation über Ausgaben}
		\begin{center}
			\includegraphics[page=31, width=0.65\linewidth, trim=30mm 27mm 27mm 60mm, clip]{\Path/resources/Vorlesung/VLSI/03_Automaten.pdf}
			\includegraphics[page=32, width=0.65\linewidth, trim=30mm 27mm 30mm 60mm, clip]{\Path/resources/Vorlesung/VLSI/03_Automaten.pdf}
		\end{center}
	
	\subsubsection{Parallele Anordnung}
		\begin{center}
			\includegraphics[page=33, width=0.65\linewidth, trim=30mm 27mm 30mm 60mm, clip]{\Path/resources/Vorlesung/VLSI/03_Automaten.pdf}
		\end{center}
		
	\subsubsection{Serielle Anordnung}
		\begin{center}
			\includegraphics[page=34, width=0.65\linewidth, trim=30mm 27mm 27mm 60mm, clip]{\Path/resources/Vorlesung/VLSI/03_Automaten.pdf}
		\end{center}
		
	\subsubsection{Flusskontrolle}
		\begin{center}
			\includegraphics[page=35, width=0.65\linewidth, trim=30mm 40mm 27mm 60mm, clip]{\Path/resources/Vorlesung/VLSI/03_Automaten.pdf}
		\end{center}
		
	\subsubsection{Beispiel: Produktautomat}
		\begin{center}
			\includegraphics[page=36, width=0.65\linewidth, trim=30mm 50mm 27mm 60mm, clip]{\Path/resources/Vorlesung/VLSI/03_Automaten.pdf}
		\end{center}
		
	\subsubsection{Beispiel: Datenverarbeitung}
		\begin{center}
			\includegraphics[page=37, width=0.65\linewidth, trim=30mm 30mm 27mm 60mm, clip]{\Path/resources/Vorlesung/VLSI/03_Automaten.pdf}
			\includegraphics[page=38, width=0.65\linewidth, trim=30mm 30mm 27mm 60mm, clip]{\Path/resources/Vorlesung/VLSI/03_Automaten.pdf}
			\includegraphics[page=39, width=0.65\linewidth, trim=30mm 30mm 27mm 60mm, clip]{\Path/resources/Vorlesung/VLSI/03_Automaten.pdf}
		\end{center}
		
	\subsubsection{Beispiel: De-/Multiplexer}
		\begin{center}
			\includegraphics[page=40, width=0.65\linewidth, trim=30mm 30mm 27mm 60mm, clip]{\Path/resources/Vorlesung/VLSI/03_Automaten.pdf}
		\end{center}

\subsection{Asynchrone Kopplung}

	\subsubsection{Abtastproblem}
	
	\begin{wrapfigure}{r}{0.25\linewidth}
  		\vspace{-20pt}
	  	\begin{center}
			\includegraphics[page=41, width=0.6\linewidth, trim=210mm 45mm 27mm 65mm, clip]{\Path/resources/Vorlesung/VLSI/03_Automaten.pdf}
	  	\end{center}
		\vspace{-20pt}
	\end{wrapfigure}
	
	Kommunikation zwischen verschiednen Taktdomänen aufgrund verschiedener Taktfrequenzen und -phasen komplexer:
	\begin{itemize}
		\item keine feste Zuordnung zwischen zwei Zeitpunkten, \\allg. Annahme \( t_n = n * T_1 \neq m* T_2 = t_m\) mit $n,m \in N$
		\item Abtastung von Signalvektoren kann zu Fehlern aufgrund verschiedener Signallaufzeiten führen
	\end{itemize}
	
	\subsubsection{Flusskontrolle}
	Berücksichtigung verschiedener Taktfrequenzen im Protokoll notwendig:
	\begin{itemize}
		\item Mehrfaches Lesen desseleben Ausgabezeichens, \\wenn $f_2 > f_1$
		\item Verpassen von Ausgabezeichen, \\wenn $f_2 < f_1$
	\end{itemize} 
	
	\subsubsection{Übertragung eines einzelnen Bits}
		\begin{center}
			\includegraphics[page=43, width=0.65\linewidth, trim=30mm 27mm 27mm 60mm, clip]{\Path/resources/Vorlesung/VLSI/03_Automaten.pdf}
		\end{center}
	
	\subsubsection{Übertragung eines Signalvektors}
	Mehrere Varainten möglich:
	\begin{enumerate}
		\item Gray-Code
		\item Steuerung mittels einzelner, seperater Bits
		\item Warteschlange mittels Cross-Clock-FIFO
	\end{enumerate}
	
	\subsubsection{Variante 1: Gray-Code}
		\includegraphics[page=45, width=0.65\linewidth, trim=30mm 27mm 27mm 60mm, clip]{\Path/resources/Vorlesung/VLSI/03_Automaten.pdf}\\
		\includegraphics[page=46, width=0.65\linewidth, trim=30mm 27mm 27mm 60mm, clip]{\Path/resources/Vorlesung/VLSI/03_Automaten.pdf}
		
	\subsubsection{Variante 2: Steuerung mittels einzelner Bits}
		\includegraphics[page=47, width=0.65\linewidth, trim=25mm 30mm 27mm 60mm, clip]{\Path/resources/Vorlesung/VLSI/03_Automaten.pdf}
		
	\subsubsection{Variante 3: FIFO-Warteschlange}
		\includegraphics[page=48, width=0.65\linewidth, trim=25mm 30mm 27mm 60mm, clip]{\Path/resources/Vorlesung/VLSI/03_Automaten.pdf}\\
		
		\includegraphics[page=49, width=0.65\linewidth, trim=25mm 30mm 27mm 60mm, clip]{\Path/resources/Vorlesung/VLSI/03_Automaten.pdf}
		
		
		
		
		
\section{Initialisierung}
	
\subsection{Reset vs. Power-Up}
	\paragraph{Nicht programmierbare Schaltkreise:}
	\begin{itemize}
		\item Reset notwendig für Initialisierung der Zustandsregister
		\item Datenregister können von Automaten initialisiert werden
	\end{itemize}
	
	\paragraph{Programmierbare Schaltkreise:}
	\begin{itemize}
		\item Initiale Registerbelegung wird durch Programmierung festgelegt
		\item Reset-Eingang ist dh. optional
	\end{itemize}
	
	\paragraph{$\rightarrow$ Wiederverwendungsgerechter Entwurf:}
	\begin{itemize}
		\item Reset-Eingang vorsehen
		\item (Zustands-)Register bei Power-Up und Reset gleichermaßen belegen
	\end{itemize}
	
\subsection{Synchrones Reset}
