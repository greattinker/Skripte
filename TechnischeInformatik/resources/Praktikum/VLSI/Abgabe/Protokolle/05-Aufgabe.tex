\section{Aufgabe 5 - HALLO-Anzeige}
\subsection{Entwurf}
	\begin{description}
	\item[zu a)] Es müssen 5 Zeichen kodiert werden (H, A, L, O, Leerzeichen).\\
			\\\(ld\, 5 = 3\)\\\\
			Daher werden für eine Binärkodierung mindestens 3 Bits benötigt.\\\\
			\begin{tabular}{|c|c|c||c|c|c|c|c|c|c|c|c|}
		\hline
		\multicolumn{3}{|c||}{Input} & \multicolumn{9}{|c|}{Output} \\
		\hline
		\textsc{BIT2} & \textsc{BIT1} & \textsc{BIT0} & 
		\textsc{CHAR} & \textsc{A} & \textsc{B} & \textsc{C} & \textsc{D} & \textsc{E} & \textsc{F} & \textsc{G} & \textsc{dot} \\ 
		\hline
		\hline
		0 & 0 & 0 &   & 1 & 1 & 1 & 1 & 1 & 1 & 1 & 1 \\ 
		0 & 0 & 1 & H & 1 & 0 & 0 & 1 & 0 & 0 & 0 & 1 \\
		0 & 1 & 0 & A & 0 & 0 & 0 & 1 & 0 & 0 & 0 & 1 \\
		0 & 1 & 1 & L & 1 & 1 & 1 & 0 & 0 & 0 & 1 & 1 \\
		1 & 0 & 0 & O & 0 & 0 & 0 & 0 & 0 & 0 & 1 & 1 \\
		\hline
	\end{tabular}
	\item[b)] Für das Schieberegister ist der Zählerzustand ein Enable-Signal
	\item[c)]  (\ref{lst:05-hallo}~Hallo.vhdl-Code) 
	\end{description}

	

\subsection{Auswertung}
	\paragraph{Ressourcenbedarf}
	\begin{itemize} 
	\item 73 Logik-Elemente
	\item 61 Register
	\item 34 Pins 
	\item maximale Taktfrequenz von 262 MHz
	\end{itemize}
