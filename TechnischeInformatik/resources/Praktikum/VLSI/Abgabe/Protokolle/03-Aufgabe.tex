\section{Aufgabe 3 - Modulo-n-Zähler}
\subsection{Entwurf}
\begin{description}
\item[a)] Der Zähler ist nach 50 Millionen Schritten zurückzusetzen (50 MHz Takt entspricht 50 Millionen Taktperioden pro Sekunde)
\item[b)] Für das Schieberegister ist der Zählerzustand ein Enable-Signal
\item[c)] \hfill 
	\begin{description}
	\item[Input] \hfill 
		\begin{itemize}
			\item 50MHz Takt
			\item Reset (Schiebeschalter SW0)
		\end{itemize}
	\item[Output] LED-Zeile
	\item[Ansatz] 2 Komponenten: Schieber und Zähler 
	\end{description}	
\end{description}
\paragraph{Zähler} gibt alle 50-Millionen Taktperioden (50MHz Takt ergibt $50\cdot10^6$ Taktschritte pro Sekunde) einen Takt lang ein enable-Signal aus. 
(\ref{lst:03-zaehler}~Zaehler.vhdl-Code)
\paragraph{Schieber} beinhaltet den Zähler als Komponente und verschiebt bei dessen enable-Signal die LED-Anzeige um eine Stelle pro Takt. (\ref{lst:03-schieber}~Schieber.vhdl-Code)

\subsection{Auswertung}
\paragraph{Ressourcenbedarf}
\begin{itemize} 
\item 60 Logik-Elemente
\item davon 38 dedizierte Logik-Elemente
\item 12 Pins 
\item maximale Taktfrequenz von 250 MHz
\end{itemize}
