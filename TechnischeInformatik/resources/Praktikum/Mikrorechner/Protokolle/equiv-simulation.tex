\subsection{Äquivalenzprüfung duch Simulation}

\subsubsection{Voraussetzungen}
\begin{itemize}
	\item Schaltkreise müssen gleichviele Eingänge bzw. Ausgänge besitzen
	\item falls in einem Schaltkreis ein Eingang bzw. Ausgang vorkommt, muss ein gleichnamiger Eingang bzw. Ausgang auch in den anderen Schaltkreisen vorkommen
	\item es können beliebig viele Schaltkreise verglichen werden, falls einer nicht äquivalent mit einem anderen ist, wird \code{false} zurückgegeben
	\item ist kein Schaltkreis definiert, wird \code{false} zurückgegeben
	\item ist nur ein Schaltkreis definiert, wird \code{true} zurückgegeben

\end{itemize}

\subsubsection{Vorgehen}
\begin{enumerate}
	\item über mögliche Zustände ($2^N$ mit N ... Anzahl der Eingänge) iterieren

	\begin{enumerate}
		\item durch Schaltkreise iterieren
		\begin{enumerate}
			\item Zustand auf Eingänge abbilden
			\item Schaltkreis traversieren und Ausgangsbelegung ermitteln
		\end{enumerate}
		\item Belegungen mit denen des vorangegangenen Schaltkreises Vergleichen
		\item Bei unterschiedlicher Belegung wird \code{false} zurückgegeben
		\item Sonst wird die Iteration über die Zustände fortgesetzt
	\end{enumerate}
	
	\item wurde über alle $2^N$ Zustände iteriert und keine Varianz der Ausgangsbelegung bei den Schaltkreisen festgestellt, wird \code{true} zurückgegeben
\end{enumerate}


\subsubsection{Implementierung}
Für alle Circuits werden die möglichen Inputs berechnet und die erhaltenen Outputs miteinander verglichen. Sollte dabei ein Circuit enthalten sein, der abweicht, wird false zurückgegeben (Circuits sind nicht äquivalent). 
