\section{Äquivalenzprüfung durch SAT-Solver}

\subsection{Voraussetzungen}
\begin{itemize}
	\item Schaltkreise müssen gleichviele Eingänge bzw. Ausgänge besitzen
	\item falls in einem Schaltkreis ein Eingang bzw. Ausgang vorkommt, muss ein gleichnamiger Eingang bzw. Ausgang auch in den anderen Schaltkreisen vorkommen
	\item es können beliebig viele Schaltkreise verglichen werden, falls einer nicht äquivalent mit einem anderen ist, wird \code{false} zurückgegeben
	\item ist kein Schaltkreis definiert, wird \code{false} zurückgegeben
	\item ist nur ein Schaltkreis definiert, wird \code{true} zurückgegeben

\end{itemize}


\subsection{Vorgehen}
\begin{enumerate}
	\item es wird über alle Schaltkreise iteriert
	
	\begin{enumerate}
		\item die KNFs aller Gatter werden ermittelt
		\item falls der Schaltkreis nicht der erste ist, werden seine Eingänge mit denen des ersten Schaltkreises verbunden d.h. die Identität wird duch eine KNF dargestellt
		\item falls der Schaltkreis nicht der erste ist, werden seine Ausgänge XOR mit denen des ersten Schaltkreises verbunden d.h. als KNF dargestellt
	\end{enumerate}
	
	\item alle Ausgänge der XOR-Verbindungen werden OR verküpft, d.h. es wird eine Klausel gebildet, die alle diese Ausgänge enthält
	\subitem lässt sich eine dieser Variablen auf \code{true} abbilden (die These ist SATISFIABLE), dann sind die definierten Schaltkreise nicht äquivalent 
	\item gesamte KNF wird in Datei geschrieben und der SATsolver gestartet
\end{enumerate}


\subsection{Implementierung}
Der verwendete SAT-Solver ist miniSAT in der Version 1.14 als Binary. Die generierten KNF-Klauseln werden in eine Datei mit dem Namen cnf geschrieben und anschliesend vom SAT-Solver gelesen. Die jeweiligen Gatter-Klassen sind so implementiert, dass sie ihre Funktion als KNF darstellen können. Die Klasse \code{Solver} enthält einen Zähler für die Benennung der KNF-Variablen.
