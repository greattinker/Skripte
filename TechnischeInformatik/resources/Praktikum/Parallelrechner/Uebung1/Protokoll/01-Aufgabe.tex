\section{Aufgabe 1 - Code-Optimierung}

%\subsection{Aufgabenstellung}
%Für die Multiplikation zweier nxn-Matrizen soll ein möglichst effizienter Algorithmus
%gefunden werden. Nutzen Sie dazu den vorgegebenen Quelltext, der bereits die
%Basisvariante und eine Zeitmessroutine enthält. Diese Basisvariante sollen Sie
%optimieren – zunächst ohne zusätzliche Compiler-Flags.


\label{lst:matmul0-orig}
\lstinputlisting[caption={matmul0.c-Original}, language=C, linerange=47-59, firstnumber=47]{../matmul0-original.c}


\subsection{Optimierungen}
\begin{enumerate}
	\item Speichern öfters genutzter, zusammengesetzter Werte \\
		\code{
			i\_mult\_dim = i * dim; \\
				i\_mult\_dim\_add\_k = i\_mult\_dim + k; \\
				k\_mult\_dim = k * dim;
			}
				
	\item halten des Wertes einer Matrix für Multiplikation (loop-interchange)\\
		 \code{ A{[} i\_mult\_dim\_add\_k {]} }
	
\end{enumerate}

\label{lst:matmul0}
\lstinputlisting[caption={matmul0.c-optimiert}, language=C, linerange=46-63, firstnumber=46]{../matmul0.c}
